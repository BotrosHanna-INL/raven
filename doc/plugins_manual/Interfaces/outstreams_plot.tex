\section{Plots}
\label{sec:outstreams_plot}
New plots that are specific to particular applications are possible through \xmlNode{OutStreams}
\xmlNode{Plot} plugins.

These plotting plugins should inherit from the \texttt{PlotPlugin} base class defined in
\begin{lstlisting}[language=bash]
  raven/framework/PluginBaseClasses/OutStreamPlotPlugin.py
\end{lstlisting}
which sets up the plotting tool to be found when RAVEN runs.

A good example of the \texttt{PlotPlugin} can be found in the RAVEN ExamplePlugin, found at
\begin{lstlisting}[language=bash]
  raven/plugins/ExamplePlugin/src/CorrelationPlot.py
\end{lstlisting}

There are a few methods that can or must be implemented for new plotting strategies.

%
%
\subsection{\texttt{run} method, required}
The \texttt{run} method is the primary execution method for the PlotPlugin. Here, whatever data
handling and plotting mechanics will be executed. Note that \texttt{run} does not receive any inputs;
often, the source for the data to be plotted will be identified in the \texttt{initialize} method.

The \texttt{run} method can perform many actions including data manipulation, creation of
\texttt{matplotlib} figures and axes, saving figures to file, and so forth. In the end, \texttt{run}
should not return anything.

%
%
\subsection{Constructor, optional}
As the constructor for Python classes, the \texttt{\_\_init\_\_} method should be extended to define
any instance variables used in the plugin class. If no instance variables are used, this may be ommitted.

Any call to \texttt{\_\_init\_\_} must include a call to the parent's constructor, such as
\begin{lstlisting}[language=python]
  def __init__(self):
    """ ... """
    super().__init__()
\end{lstlisting}
This assures access to basic RAVEN functionalities required to use the plugin.

%
%
\subsection{Input Handling methods, optional}
The class method \texttt{getInputSpecification} and corresponding instance method
\texttt{handleInput} are how RAVEN determines what user inputs are allowed and how those input
values get stored in the plugin. Acceptable inputs are defined in \texttt{getInputSpecification}
and then read in during \texttt{handleInput}. Both of these methods require a call to \texttt{super}
to function as expected.

%
%
\subsection{Initialization, optional}
The aptly-named \texttt{initialize} method is used at the start of every RAVEN \xmlNode{Step} to prepare
for execution. A common task for this method is to find the source of the data to plot. To make this
process easier, RAVEN provides a \texttt{self.findSource} method that can search the input dictionary
provided to \texttt{initialize} and find a \xmlNode{DataObject} by string name.

%
%
