%%%%%%%%%%%%%%%%%%%%%%%%%%%%%%%
%%% AccelerateCFD Interface %%%
%%%%%%%%%%%%%%%%%%%%%%%%%%%%%%%
\subsection{AccelerateCFD Interface}
\label{subsec:AccelerateCFDInterface}
This section covers the input specification for running AccelerateCFD
 (\url{https://github.com/IllinoisRocstar/AccelerateCFD_CE})
 through RAVEN. It is important to notice
that this explanation assumes that the user already knows how to use AccelerateCFD. The existing inteface can be used to perturb any parameter
contained in the AccelerateCFD input file (i.e., podInputs.xml).

\subsubsection{Files}
In the \xmlNode{Files} section, as specified before, all the files needed
for the code to run should be specified. In the case of AccelerateCFD,
the files needed are the following:
\begin{itemize}
  \item AccelerateCFD input file ``podInputs.xml''. This input file must be tagged with
          the $type="input"$
  \item Mesh file for x coordinates'. This input file must be tagged with
          the $type="mesh-x"$
  \item Mesh file for y coordinates'. This input file must be tagged with
          the $type="mesh-y"$
  \item Mesh file for z coordinates'. This input file must be tagged with
          the $type="mesh-z"$
\end{itemize}
%
Example:
\begin{lstlisting}[style=XML]
  <Files>
    <Input name="AcceleratedCFDinput" type="input">podInputs.xml</Input>
    <Input name="Cx" type="mesh-x">Cx</Input>
    <Input name="Cy" type="mesh-y">Cy</Input>
    <Input name="Cz" type="mesh-z">Cz</Input>
  </Files>
\end{lstlisting}

%%%%%%%%%%%%%%%%%%%%%%%%%%%%%%%%%%%%%%%%%%%%%%%%%%%%%X
\subsubsection{Models}
In the \xmlNode{Models} block, the AccelerateCFD executable needs to be specified.
The  path to the AccelerateCFD executable must be included.
Here is a standard example of what can be used:
\begin{lstlisting}[style=XML]
  <Models>
    <Code name="myAccelerateCFD" subType="AccelerateCFD">
      <executable>./romRun</executable>
      <outputLocations>
          <coordinates>
           (0.0,0.0,0.0) (1.0,1.0,1.0) (min,min,min) (max,max,max) (middle,middle,middle)
          </coordinates>
      </outputLocations>
    </Code>
  </Models>
\end{lstlisting}

The \xmlNode{Code} XML node contains the information needed to execute the specific External Code. This
XML node accepts the following attributes:
\begin{itemize}
  \item \xmlAttr{name}, \xmlDesc{required string attribute}, user-defined identifier of this model.
    \nb As with other objects, this identifier can be used to reference this specific entity from other input
    blocks in the XML.
  \item \xmlAttr{subType}, \xmlDesc{required string attribute}, specifies the code that needs to be
    associated to this Model.
\end{itemize}
This model can be initialized with the following children:
\begin{itemize}
  \item \xmlNode{executable}, \xmlDesc{string, required field}, specifies the path of the executable to
    be used; \nb Either an absolute or relative path can be used.
\end{itemize}


%%%%%%%%%%%%%%%%%%%%%%%%%%%%%%%%%%%%%%%%%%%%%%%%%%%%%%
\subsubsection{Distributions}
The \xmlNode{Distributions} block defines the distributions that are going to be used for the sampling
of the variables defined in the \xmlNode{Samplers} block. For all the possible distributions and all
their possible inputs, please see the chapter about Distributions (see~\ref{sec:distributions}). Here is an example of a
Uniform distribution:
\begin{lstlisting}[style=XML]
  <Distributions>
    <UniformDiscrete name='uni'>
      <lowerBound>10</lowerBound>
      <upperBound>30</upperBound>
      <strategy>withoutReplacement</strategy>
    </UniformDiscrete>
  </Distributions>
\end{lstlisting}
%
%%%%%%%%%%%%%%%%%%%%%%%%%%%%%%%%%%%%%%%%%%%%%%%%%%%%%%
\subsubsection{Samplers}
The \xmlNode{Samplers} block defines the variables to be sampled. After defining a sampling scheme, the variables to be sampled and
their distributions are identified in the \xmlNode{variable} blocks.
The \xmlAttr{name} must be formatted according to the AccelerateCFD parameter name (in the XML input file)
An example of a \xmlNode{Samplers} block is given below:
\begin{lstlisting}[style=XML]
 <Samplers>
    <MonteCarlo name="myMC">
      <samplerInit>
        <limit>5</limit>
      </samplerInit>
      <variable name='velMods'>
        <distribution>uni</distribution>
      </variable>
    </MonteCarlo>
  </Samplers>
\end{lstlisting}

%%%%%%%%%%%%%%%%%%%%%%%%%%%%%%%%%%%%%%%%%%%%%%%%%%%%%%
\subsubsection{Steps}
In this section, the \xmlNode{MultiRun} will be used. As shown in the following, the AccelerateCFD
input files are listed in \xmlNode{Files} and are linked here using \xmlNode{Input}, the \xmlNode{Model}
and \xmlNode{Sampler} defined in previous sections will be used in this \xmlNode{MultiRun}. The
outputs will be saved in the \textbf{DataObject} `resultPointSet'.

\begin{lstlisting}[style=XML]
  <Steps>
    <MultiRun name="run">
      <Input   class="Files"       type="input">podInputs.xml</Input>
      <Input   class="Files"       type="mesh-x">Cx</Input>
      <Input   class="Files"       type="mesh-y">Cy</Input>
      <Input   class="Files"       type="mesh-x">Cz</Input>
      <Model   class="Models"      type="Code">AcceleratedCFDCode</Model>
      <Sampler class="Samplers"    type="MonteCarlo">myMC</Sampler>
      <Output  class="Databases"   type="HDF5">DataAcceleratedCFD</Output>
      <Output  class="DataObjects" type="HistorySet">AcceleratedCFDHistorySet</Output>
      <Output  class="DataObjects" type="PointSet">AcceleratedCFDPointSet</Output>
    </MultiRun>
  </Steps>
\end{lstlisting}

%%%%%%%%%%%%%%%%%%%%%%%%%%%%%%%%%%%%%%%%%%%%%%%%%%%%%%
\subsubsection{Output File Conversion}
The AccelerateCFD software produces CFD-grade results (velocity and scalar fields). The interface
will extract the results and deploy them using the following format:
\begin{itemize}
  \item $variableName-coordinates-axis$. For example: $volVectorField-0\_0\_0\_0\_0\_0-x$ or $volVectorField-min_min_min-x$
\end{itemize}
For example, the following example $DataObject$:
\begin{lstlisting}[style=XML]
<DataObjects>
  <PointSet name="AcceleratedCFDPointSet">
    <Input>velMods,scalMods</Input>
    <Output>time,volVectorField-0_0_0_0_0_0-x,
    volVectorField-0_0_0_0_0_0-y,
    volVectorField-0_0_0_0_0_0-z,
    volVectorField-1_0_1_0_1_0-x,
    volVectorField-1_0_1_0_1_0-y,
    volVectorField-1_0_1_0_1_0-z,
    volVectorField-min_min_min-x,
    volVectorField-min_min_min-y,
    volVectorField-min_min_min-z,
    volVectorField-max_max_max-x,
    volVectorField-max_max_max-y,
    volVectorField-max_max_max-z,
    volVectorField-middle_middle_middle-x,
    volVectorField-middle_middle_middle-y,
    volVectorField-middle_middle_middle-z
    </Output>
  </PointSet>
  <HistorySet name="AcceleratedCFDHistorySet">
    <Input>velMods,scalMods</Input>
    <Output>time,volVectorField-0_0_0_0_0_0-x,
    volVectorField-0_0_0_0_0_0-y,
    volVectorField-0_0_0_0_0_0-z,
    volVectorField-1_0_1_0_1_0-x,
    volVectorField-1_0_1_0_1_0-y,
    volVectorField-1_0_1_0_1_0-z,
    volVectorField-min_min_min-x,
    volVectorField-min_min_min-y,
    volVectorField-min_min_min-z,
    volVectorField-max_max_max-x,
    volVectorField-max_max_max-y,
    volVectorField-max_max_max-z,
    volVectorField-middle_middle_middle-x,
    volVectorField-middle_middle_middle-y,
    volVectorField-middle_middle_middle-z
    </Output>
  </HistorySet>
</DataObjects>
\end{lstlisting}
