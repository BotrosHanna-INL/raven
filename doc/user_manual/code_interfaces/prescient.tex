\subsection{Prescient Interface}
\label{subsec:prescientInterface}

\subsubsection{General Information}
The Prescient Interface is used to run the open source Prescient
production cost modeling platform available from
\url{https://github.com/grid-parity-exchange/Prescient}

This allows inputs to be perturbed and data to be read out.

\subsubsection{Sampler}

For perturbing inputs, the sampled variable needs to be placed inside
of \$( )\$ like \verb'$(var)$'. The sampled variable can have a
constant added or a multiplication factor like \verb'$(var+3.2)$' or
\verb'$(var*2.1)$' or \verb'$(var*5.0+7.0)$' or \verb'$(a*-2.0)$'
These can be placed in any of the .dat or .csv files that are listed
in the \xmlNode{Files} section as \xmlAttr{type="PrescientInput"} An
example line could be: \verb'Abel 1 $(var)$'

\begin{lstlisting}[style=XML]
  <Samplers>
    <Grid name="grid">
      <variable name="var">
        <distribution>dist</distribution>
        <grid construction="equal" steps="1" type="CDF">0.0 1.0</grid>
      </variable>
    </Grid>
  </Samplers>
\end{lstlisting}


\subsubsection{Files}

There are two types of inputs in the \xmlNode{Files} section.  The
\xmlAttr{type="PrescientRunnerInput"} ones are passed as an argument
to the \texttt{runner.py} If multiple PrescientRunnerInput files are
specified then \texttt{runner.py} will be called multiple times (which
can be used to run a populate and then simulate command).  The
\xmlAttr{type="PrescientInput"} are just used as additional inputs
that have the data in them perturbed.

\begin{lstlisting}[style=XML]
  <Files>
  <Input name="simulate" type="PrescientRunnerInput"
  >simulate_day.txt</Input>
    <Input name="structure" type="PrescientInput"
    subDirectory="scenarios/pyspdir_twostage/2020-07-10/"
    >ScenarioStructure.dat</Input>
    <Input name="scenario_1" type="PrescientInput"
    subDirectory="scenarios/pyspdir_twostage/2020-07-10/"
    >Scenario_1.dat</Input>
    <Input name="actuals" type="PrescientInput"
    subDirectory="scenarios/pyspdir_twostage/2020-07-10/"
    >Scenario_actuals.dat</Input>
    <Input name="forcasts" type="PrescientInput"
    subDirectory="scenarios/pyspdir_twostage/2020-07-10/"
    >Scenario_forecasts.dat</Input>
    <Input name="scenarios" type="PrescientInput"
    subDirectory="scenarios/pyspdir_twostage/2020-07-10/"
    >scenarios.csv</Input>
  </Files>
\end{lstlisting}

\subsubsection{Models}

The \xmlNode{Code} model can be used with the
\xmlAttr{subType="Prescient"} to run the Prescient Code Interface.
The block currently does not have any option xml nodes.

\begin{lstlisting}[style=XML]
  <Models>
    <Code name="TestPrescient" subType="Prescient">
      <executable>
      </executable>
    </Code>
  </Models>
\end{lstlisting}

\subsubsection{Output Files Conversion}

The code interface reads in the \texttt{hourly\_summary.csv} and the
\texttt{bus\_detail.csv} files. It will generate a \texttt{Date\_Hour}
variable that can be used as the \xmlNode{pivotParameter} and is a
string with the date and hour. It also generates an \texttt{Hour}
variable that is the hour as an integer.  From the hourly summary it
will generate variables like TotalCosts and the other data that
appears there.  For each of the busses in the bus detail file it
generates variables like \texttt{Clay\_LMP} that can be used.

Exactly which variables will appear will vary depending on the
Prescient input files, but typical ones include \texttt{TotalCosts},
\texttt{FixedCosts}, \texttt{VariableCosts}, \texttt{LoadShedding},
\texttt{OverGeneration}, \texttt{ReserveShortfall},
\texttt{RenewablesUsed}, \texttt{RenewablesCurtailment},
\texttt{Demand}, \texttt{Price}, and \texttt{NetDemand}. Variables
that can be included for a typical bus could include ones like
\texttt{Abel\_LMP}, \texttt{Abel\_LMP\_DA}, \texttt{Abel\_Shortfall},
and \texttt{Abel\_Overgeneration}.

\begin{lstlisting}[style=XML]
    <HistorySet name="samples">
      <Input>var</Input>
      <Output>Date_Hour, TotalCosts, FixedCosts, VariableCosts, LoadShedding, OverGeneration, ReserveShortfall, RenewablesUsed, RenewablesCurtailment, Demand, Price, NetDemand, Abel_LMP, Clay_LMP </Output>
      <options>
        <pivotParameter>Date_Hour</pivotParameter>
      </options>
    </HistorySet>
\end{lstlisting}


\subsubsection{Installation of Libraries}

Installing Prescient so that RAVEN can run it requires that RAVEN and
Prescient have a superset of the libraries that they use so that both
can run.  One way to set this up is to install RAVEN, and then source
the conda load script and inside of the conda raven libraries
environment do the Prescient and Egret install.  This is shown in the
following listing:

\begin{lstlisting}
  #first clone raven, Egret and Prescient into a directory
  git clone git@github.com:idaholab/raven.git
  git clone git@github.com:grid-parity-exchange/Prescient.git
  git clone git@github.com:grid-parity-exchange/Egret.git
  #Switch to raven directory
  cd raven
  #install raven libraries
  ./scripts/establish_conda_env.sh --install
  #switch to using raven libraries
  source ./scripts/establish_conda_env.sh --load
  #Switch to Prescient and install
  cd ../Prescient
  python setup.py develop --user
  conda install -c conda-forge coincbc
  #Switch to Egret and install
  cd ../Egret/
  pip install --user -e .
\end{lstlisting}

Note that the path to \texttt{runner.py} may need to be added to the PATH variable via a command like: \verb'PATH="$PATH:$HOME/.local/bin"'

