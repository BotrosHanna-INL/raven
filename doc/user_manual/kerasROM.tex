%%%%%% command used for TensorFlow-Keras Deep neural networks %%%%%%%%%%%%%%%%%
\newcommand{\layerNameAttr}[0]
{
    This node require the following attribute:
    \begin{itemize}
      \item \xmlAttr{name}, \xmlDesc{string, required field}, name of this layer. The value will be
        used in \xmlNode{layer\_layout} to construct the fully connected neural network.
    \end{itemize}
}
%%% Arguments for Dense Layer %%%
\newcommand{\activation}[0]
{
      \item \xmlNode{activation}, \xmlDesc{string, optional field}, including
        {`relu', `tanh', `elu', `selu', `softplus', `softsign', `sigmoid', `hard\_sigmoid', `linear', `softmax'}.
        (see~\ref{activationsDNN})
        \default{linear}
}
\newcommand{\dimOut}[0]
{
      \item \xmlNode{dim\_out}, \xmlDesc{positive integer, required except if this layer is used as the last output layer},
        dimensionality of the output space of this layer
}
\newcommand{\useBias}[0]
{
  \item \xmlNode{use\_bias}, \xmlDesc{boolean, optional field}, whether the layer uses a bias vector.
    \default{True}
}
\newcommand{\kernelInitializer}[0]
{
  \item \xmlNode{kernel\_initializer}, \xmlDesc{string, optional field}, initializer for the kernel weights matrix
    (see~\ref{initializersDNN}).
    \default{glorot\_uniform}
}
\newcommand{\biasInitializer}[0]
{
  \item \xmlNode{bias\_initializer}, \xmlDesc{string, optional field}, intializer for the bias vector
    (see ~\ref{initializersDNN}).
    \default{zeros}
}
\newcommand{\kernelRegularizer}[0]
{
  \item \xmlNode{kernel\_regularizer}, \xmlDesc{string, optional field}, regularizer function applied to
    the kernel weights matrix (see ~\ref{regularizersDNN}).
    \default{None}
}
\newcommand{\biasRegularizer}[0]
{
  \item \xmlNode{bias\_regularizer}, \xmlDesc{string, optional field}, regularizer function applied to the bias vector
    (see~\ref{regularizersDNN}).
    \default{None}
}
\newcommand{\activityRegularizer}[0]
{
  \item \xmlNode{activity\_regularizer}, \xmlDesc{string, optional field}, regularizer function applied to the output
    of the layer (its "activation"). (see~\ref{regularizersDNN})
    \default{None}
}
\newcommand{\kernelConstraint}[0]
{
  \item \xmlNode{kernel\_constraint}, \xmlDesc{string, optional field}, constraint function applied to the kernel weights
    matrix (see~\ref{constraintsDNN}).
    \default{None}
}
\newcommand{\biasConstraint}[0]
{
  \item \xmlNode{bias\_constraint}, \xmlDesc{string, optional field}, constraint function applied to the bias vector
    (see ~\ref{constraintsDNN})
    \default{None}
}
%%% Arguments for Dropout Layer %%%
\newcommand{\rate}[0]
{
      \item \xmlNode{rate}, \xmlDesc{float between 0 and 1, optional field}, fraction of the input units to drop.
        \default{0}
}
\newcommand{\noiseShape}[0]
{
      \item \xmlNode{noise\_shape}, \xmlDesc{list of integers, optional field}, 1D integer tensor representing the shape
        of the binary dropout mask that will be multiplied with the input.
        \default{None}
}
\newcommand{\seed}[0]
{
      \item \xmlNode{seed}, \xmlDesc{integer, optional field}, a integer to use as random seed.
        \default{None}
}
%%% Arguments for LSTM Layer %%%
\newcommand{\recurrentActivation}[0]
{
      \item \xmlNode{recurrent\_activation}, \xmlDesc{string, optional field}, activation function to use for the recurrent
        step, including {`relu', `tanh', `elu', `selu', `softplus', `softsign', `sigmoid', `hard\_sigmoid', `linear', `softmax'}.
        \default{hard\_sigmoid}
}
\newcommand{\recurrentInitializer}[0]
{
      \item \xmlNode{recurrent\_initializer}, \xmlDesc{string, optional field}, used for the linear transformation of
        the recurrent state (see ~\ref{initializersDNN}).
        \default{orthogonal}
}
\newcommand{\unitForgetBias}[0]
{
      \item \xmlNode{unit\_forget\_bias}, \xmlDesc{boolean, optional field}, add 1 to the bias of the forget gate at
        initialization if True.
        \default{True}
}
\newcommand{\recurrentRegularizer}[0]
{
      \item \xmlNode{recurrent\_regularizer}, \xmlDesc{string, optional field}, regularizer function applied to the
        \textit{recurrent\_kernel} weights matrix (see ~\ref{regularizersDNN}).
        \default{None}
}
\newcommand{\recurrentConstraint}[0]
{
      \item \xmlNode{recurrent\_constraint}, \xmlDesc{string, optional field}, constraint function applied to the
        \textit{recurrent\_kernel} weights matrix (see ~\ref{constraints}).
        \default{None}
}
\newcommand{\dropout}[0]
{
      \item \xmlNode{dropout}, \xmlDesc{float between 0 and 1, optional field}, fraction of the units to drop for the linear
        transformation of the inputs
        \default{0}
}
\newcommand{\recurrentDropout}[0]
{
      \item \xmlNode{recurrent\_dropout}, \xmlDesc{float between 0 and 1, optional field}, fraction of the units to drop for the linear
        transformation of the recurrent state.
        \default{0}
}
\newcommand{\returnSequence}[0]
{
      \item \xmlNode{return\_sequence}, \xmlDesc{boolean, optional field}, whether to return the last output in the output sequence, or
        full sequence.
        \default{False}
}
\newcommand{\implementation}[0]
{
      \item \xmlNode{implementation}, \xmlDesc{integer, optional field},
        implementation mode, either 1 or 2. Mode 1 will structure its operations as a larger number of smaller dot products and additions,
        whereas mode 2 will batch them into fewer, larger operations. These modes will have different performance profiles on different
        hardware and for different applications.
        \default{1}
}
\newcommand{\returnState}[0]
{
      \item \xmlNode{return\_state}, \xmlDesc{boolean, optional field},
        whether to return the last output in the output sequence, or the full sequence.
        \default{False}
}
\newcommand{\goBackwards}[0]
{
      \item \xmlNode{go\_backwards}, \xmlDesc{boolean, optional field},
        if True, process the input sequence backwards and return the reversed sequence.
        \default{False}
}
\newcommand{\stateful}[0]
{
      \item \xmlNode{stateful}, \xmlDesc{boolean, optional field},
        if True, the last state for each sample at index i in a batch will be used as initial state for the sample
        of index i in the following batch.
        \default{False}
}
\newcommand{\unroll}[0]
{
      \item \xmlNode{unroll}, \xmlDesc{boolean, optional field},
        if True, the network will be unrolled, else a symbolic loop will be used. Unrolling can speed-up a RNN,
        although it tends to be more memory-intensive. Unrolling is only suitable for short sequences.
        \default{False}
}

%%% Arguments for Conv1D Layer %%%

\newcommand{\kernelSize}[0]
{
      \item \xmlNode{kernel\_size}, \xmlDesc{integer or list of integers, required field}, specifying the length
        of the 1D convolution window.
}
\newcommand{\strides}[0]
{
      \item \xmlNode{strides}, \xmlDesc{integer or list of integers, optional field}, pecifying the stride
        length of the convolution. Specifying any stride value not equal 1 is incompatible with specifying any
        dilation\_rate value not equal 1.
        \default{1}
}
\newcommand{\padding}[0]
{
      \item \xmlNode{padding}, \xmlDesc{string, optional field},
        one of "valid", "causal" or "same" (case-insensitive).  "valid" means "no padding".
        "same" results in padding the input such that the output has the same length as the original input.
        "causal" results in causal (dilated) convolutions, e.g. output[t] does not depend on input[t + 1:].
        A zero padding is used such that the output has the same length as the original input. Useful when
        modeling temporal data where the model should not violate the temporal order.
        \default{valid}
}
\newcommand{\dataFormat}[0]
{
      \item \xmlNode{data\_format}, \xmlDesc{string, optional field},
        A string, one of "channels\_last" (default) or "channels\_first". The ordering of the dimensions in the inputs.
        "channels\_last" corresponds to inputs with shape  (batch, steps, channels) (default format for temporal data
        in Keras) while "channels\_first" corresponds to inputs with shape (batch, channels, steps).
        \default{channels\_last}
}
\newcommand{\dilationRate}[0]
{
      \item \xmlNode{dilation\_rate}, \xmlDesc{integer or list of integers, optional field},
        specifying the dilation rate to use for dilated convolution. Currently, specifying any dilation\_rate value
        not equal 1 is incompatible with specifying any strides value not equal 1.
        \default{1}
}

%%% Arguments for Pooling Layer %%%

\newcommand{\poolSize}[0]
{
      \item \xmlNode{pool\_size}, \xmlDesc{integer, required field}, size of the max pooling windows.
        \default{2}
}

\newcommand{\DenseLayer}[0]
{
  \item \xmlNode{Dense}, \xmlDesc{required field}, regular densely-connected neural network layer.
    \layerNameAttr
    In addition, this node also accepts the following subnodes
    \begin{itemize}
        \activation
        \dimOut
        \useBias
        \kernelInitializer
        \biasInitializer
        \kernelRegularizer
        \biasRegularizer
        \activityRegularizer
        \kernelConstraint
        \biasConstraint
    \end{itemize}
}
\newcommand{\DropoutLayer}[0]
{
  \item \xmlNode{Dropout}, \xmlDesc{optional field}, applies Dropout to the input. Dropout consists in
    randomly setting a fraction \xmlNode{rate} of input units to 0 at each update during training time,
    which helps prevent overfitting.
    \layerNameAttr
    In addition, this node also accepts the following subnode
    \begin{itemize}
        \rate
        \noiseShape
        \seed
    \end{itemize}
}

\newcommand{\PoolingLayer}[1]
{

  \item \xmlNode{#1}, \xmlDesc{optional field},
    In addition, this node also accepts the following subnodes
    \begin{itemize}
        \poolSize
        \strides
        \padding
        \dataFormat
    \end{itemize}
}
\newcommand{\GlobalPoolingLayer}[1]
{

  \item \xmlNode{#1}, \xmlDesc{optional field},
    In addition, this node also accepts the following subnodes
    \begin{itemize}
        \dataFormat
    \end{itemize}
}

%%%%%%%%%%%%%%%%%%%%%%%%%%%%%%%%%%%%%%%%%%%%%%%%%%%%%%%%%%%%%%%%%%%%%%%%%%%%%%%

%%%%% ROM Model - TensorFlow-Keras Interface  %%%%%%%
\subsubsection{TensorFlow-Keras Deep Neural Networks}
\label{subsubsec:TFK_DNNs}

\textcolor{red}{\\It is important to NOTE that Python3 is required in order to use these deep neural networks.
If python2 is installed, these ROMs will not be imported by RAVEN, and an error will be raised if the user tries
to use these capabilities.}

\textbf{TensorFlow} is an open source software library for high performance numerical computation. Its flexible architecture
allows easy deployment of computation across a variety of platforms (CPUs, GPUs, TPUs), and from desktops to clusters
of servers to mobile and edge devices. Originally developed by researchers and engineers from the Google Brain team
within Google’s AI organization, it comes with strong support for machine learning and deep learning and the flexible
numerical computation core is used across many other scientific domains.

\textbf{Keras} is a high-level API to build and train deep learning models. It's used for fast prototyping, advanced research,
and production, with three key advantages:
\begin{itemize}
  \item \textit{User friendly}: Keras has a simple, consistent interface optimized for common use cases.
    It provides clear and actionable feedback for user errors.
  \item \textit{Modular and composable}: Keras models are made by connecting configurable building blocks together,
    with few restrictions.
  \item \textit{Easy to extend}: Write custom building blocks to express new ideas for research. Create new layers,
    loss functions, and develop state-of-the-art models.
\end{itemize}

\textbf{tf.keras} is TensorFlow's implementation of the Keras API specification. This is a high-level API to build and train
models that include first-class support for TensorFlow-specific functionality, such as eager \textit{execution},
\textit{tf.data} pipelines, and \textit{Estimators}. \textbf{tf.keras} makes TensorFlow easier to use without sacrificing
flexibility and performance. RAVEN will utilize this high-level API to build and train deep neural networks (DNNs) as ROMs, and
these ROMs can be employed by other RAVEN entities to perform uncertainty quantification, model opimization and data analysis.

Before analyzing each classifier in detail, it is important to mention that each type has a similar syntax. In the
example below, the subnodes that can be included in the main XML node \xmlNode{ROM} are reported:
\textbf{Example:}
\begin{lstlisting}[style=XML,morekeywords={name,subType}]
<Simulation>
  ...
  <Models>
    ...
    <ROM name='aUserDefinedName' subType='whatever'>
      <Features>X,Y</Features>
      <Target>Z</Target>
      <loss>mean_squared_error</loss>
      <metrics>accuracy</metrics>
      <batch_size>4</batch_size>
      <epochs>4</epochs>
      <num_classes>2</num_classes>
      <validation_split>0.25</validation_split>
      <optimizerSetting>
        <optimizer>Adam</optimizer>
        ...
      </optimizerSetting>
      <WhateverLayer1 name="layerName1">
        ...
      </WhateverLayer1>
      ...
      <WhateverLayerN name="layerNameN">
        ...
      </WhateverLayerN>
      <layer_layout>layerName1, ..., layerNameN</layer_layout>
    </ROM>
    ...
  </Models>
  ...
</Simulation>
\end{lstlisting}

As shown in above example, in addition to the common subnodes \xmlNode{Target} and \xmlNode{Features}, the \xmlNode{ROM} of DNNs
can be initialized with the following children:
\begin{itemize}
  \item \xmlNode{loss}, \xmlDesc{string or comma separated string, optional field}, if the model has multiple outputs, you can use a different
    loss metric on each output by passing a list of loss metrics. The value that will be minimized by the model will then
    be the sum of all individual value from each loss metric. Available loss functions include \textit{mean\_squared\_error},
    \textit{mean\_absolute\_error}, \textit{mean\_absolute\_percentage\_error}, \textit{mean\_squared\_logarithmic\_error},
    \textit{squared\_hinge}, \textit{hinge}, \textit{categorical\_hinge}, \textit{logcosh}, \textit{categorical\_crossentropy},
    \textit{sparse\_categorical\_crossentropy}, \textit{binary\_crossentropy}, \textit{kullback\_leibler\_divergence},
    \textit{poisson}, \textit{cosine\_proximity}.
  \default{mean\_squared\_error}
  \item \xmlNode{metrics}, \xmlDesc{string or comma separated string, optional field}, list of metrics to be evaluated by
    the model during training and testing. available metrics include
    \textit{binary\_accuracy}, \textit{categorical\_accuracy}, \textit{sparse\_categorical\_accuracy},
    \textit{top\_k\_categorical\_accuracy}, \textit{sparse\_top\_k\_categorical\_accuracy}.
  \default{accuracy}
  \item \xmlNode{batch\_size}, \xmlDesc{integer, optional field}, number of samples per gradient update.
  \default{20}
  \item \xmlNode{epochs}, \xmlDesc{integer, optional field}, number of epochs to train the model. An epoch
    is an iteration over the entire training data.
  \default{20}
  \item \xmlNode{num\_classes}, \xmlDesc{positive integer, optional field}, dimensionality of the output space of given classifier.
  \default{1}
  \item \xmlNode{validation\_split}, \xmlDesc{float between 0 and 1, optional field}, fraction of the training data to
    be used as validation data.
  \default{0.25}
  \item \xmlNode{plot\_model}, \xmlDesc{boolean, optional field}, if true the DNN model constructed by RAVEN will be
    plotted and stored in the working directory. The file name will be \textit{"ROM name" + "\_" + "model.png"}.
    \nb This capability requires the following libraries, i.e. pydot-ng and graphviz to be installed.
  \default{False}
  \item \xmlNode{optimizerSetting}, \xmlDesc{optional field}, including several subnode depending on the type of
    optimizers.
    \begin{itemize}
      \item \xmlNode{optimizer}, \xmlDesc{string, optional field}, name of optimizer.
    \end{itemize}
    \default{Adam}
    \nb The users can also choose different optimizers to train the ROM. The default algorithm is \textit{Adam}.
    Other available optimizers include:
    \textit{SGD}, \textit{RMSprop}, \textit{Adagrad}, \textit{Adadelta}, \textit{Adamx}, \textit{Nadam}.
    For the detailed information, i.e. the parameters for each optimization, the user can refer to
    \url{https://keras.io/optimizers/}. In raven, the user can use \xmlNode{optimizerSetting} to set the
    parameters of the above optimizer as follows:
    \begin{itemize}
      \item \textbf{Adam}, adam optimizer
        \begin{itemize}
          \item \xmlNode{beta\_1}, \xmlDesc{float, optional field}, $0 < beta < 1$. Generally close to 1.
          \default{0.9}
          \item \xmlNode{beta\_2}, \xmlDesc{float, optional field}, $0 < beta < 1$. Generally close to 1.
          \default{0.999}
          \item \xmlNode{epsilon}, \xmlDesc{float, optional field}, fuzz factor.
          \default{None}
          \item \xmlNode{decay}, \xmlDesc{float, optional field}, learning rate decay over each update.
          \default{0.0}
          \item \xmlNode{lr}, \xmlDesc{float, optional field}, learning rate.
          \default{0.001}
        \end{itemize}
    %
      \item \textbf{SGD}, stochastic gradient descent optimizer.
        \begin{itemize}
          \item \xmlNode{momentum}, \xmlDesc{float, optional field}, $> 0$. Parameter that accelerates SGD in
            the relevant direction and dampens oscillations.
          \default{0.0}
          \item \xmlNode{nesterov}, \xmlDesc{boolean, optional field}, whether to apply Nesterov momentum
          \default{False}
          \item \xmlNode{decay}, \xmlDesc{float, optional field}, learning rate decay over each update.
          \default{0.0}
          \item \xmlNode{lr}, \xmlDesc{float, optional field}, learning rate.
          \default{0.001}
        \end{itemize}
    %
      \item \textbf{RMSprop}, RMSProp optimizer.
        \begin{itemize}
          \item \xmlNode{rho}, \xmlDesc{float, optional field}, $> 0$.
          \default{0.9}
          \item \xmlNode{decay}, \xmlDesc{float, optional field}, learning rate decay over each update.
          \default{0.0}
          \item \xmlNode{lr}, \xmlDesc{float, optional field}, learning rate.
          \default{0.001}
          \item \xmlNode{epsilon}, \xmlDesc{float, optional field}, fuzz factor.
          \default{None}
        \end{itemize}
    %
      \item \textbf{Adagrad}, Adagrad optimizer.
        \begin{itemize}
          \item \xmlNode{decay}, \xmlDesc{float, optional field}, learning rate decay over each update.
          \default{0.0}
          \item \xmlNode{lr}, \xmlDesc{float, optional field}, learning rate.
          \default{0.01}
          \item \xmlNode{epsilon}, \xmlDesc{float, optional field}, fuzz factor.
          \default{None}
        \end{itemize}
    %
      \item \textbf{Adadelta}, Adadelta optimizer.
        \begin{itemize}
          \item \xmlNode{decay}, \xmlDesc{float, optional field}, learning rate decay over each update.
          \default{0.0}
          \item \xmlNode{lr}, \xmlDesc{float, optional field}, learning rate.
          \default{1.0}
          \item \xmlNode{epsilon}, \xmlDesc{float, optional field}, fuzz factor.
          \default{None}
          \item \xmlNode{rho}, \xmlDesc{float, optional field}, $> 0$.
          \default{0.95}
        \end{itemize}
    %
      \item \textbf{Adamax}, Adamax optimizer
        \begin{itemize}
          \item \xmlNode{beta\_1}, \xmlDesc{float, optional field}, $0 < beta < 1$. Generally close to 1.
          \default{0.9}
          \item \xmlNode{beta\_2}, \xmlDesc{float, optional field}, $0 < beta < 1$. Generally close to 1.
          \default{0.999}
          \item \xmlNode{epsilon}, \xmlDesc{float, optional field}, fuzz factor.
          \default{None}
          \item \xmlNode{decay}, \xmlDesc{float, optional field}, learning rate decay over each update.
          \default{0.0}
          \item \xmlNode{lr}, \xmlDesc{float, optional field}, learning rate.
          \default{0.002}
        \end{itemize}
    %
      \item \textbf{Nadam},
        \begin{itemize}
          \item \xmlNode{beta\_1}, \xmlDesc{float, optional field}, $0 < beta < 1$. Generally close to 1.
          \default{0.9}
          \item \xmlNode{beta\_2}, \xmlDesc{float, optional field}, $0 < beta < 1$. Generally close to 1.
          \default{0.999}
          \item \xmlNode{epsilon}, \xmlDesc{float, optional field}, fuzz factor.
          \default{None}
          \item \xmlNode{lr}, \xmlDesc{float, optional field}, learning rate.
          \default{0.002}
        \end{itemize}
  \end{itemize}
  \item \xmlNode{layer\_layout}, \xmlDesc{comma seperated string, required}, the layout/order of layers in the
    deep neural networks. The values in the subnode should be the name of layers defined in layer node, such as
    \xmlNode{Dense}, \xmlNode{Dropout}, and \xmlNode{Conv1D}.
\end{itemize}

\nb The descriptions regarding the \xmlNode{WhateverLayer} node will be introduced in following subsections.
Basically, different classifiers will require different layers.
In addition, most core layers will accept the \xmlNode{activation} subnode (see ~\ref{activationsDNN}).

%%%%% Activation Functions  %%%%%%%
\paragraph{Activation Functions}
\label{activationsDNN}
Activations can either be used through an \xmlNode{Activation} layer, or through the
\xmlNode{activation} argument supported by all forward layers.
Available activations include:
\begin{itemize}
  \item \textit{relu}, the rectified linear unit function, returns $f(x) = max(0, x)$.
  \item \textit{tanh}, the hyperbolic tan function, returns $f(x) = tanh(x)$.
  \item \textit{elu}, exponential linear units try to make the mean activations closer to zero which speeds
    up learning. $f(x) = x$ if $x \ge 0$, otherwise $(exp(x) - 1.)$.
  \item \textit{selu}, scaled exponential linear unit, i.e. $scale * elu(x, alpha)$, where $scale, alpha$
    are pre-defined constants.
  \item \textit{softplus}, a smooth approximation to the rectifier linear unit function, return
    $f(x) = log(1. + exp(x))$.
  \item \textit{softsign}, return $f(x) = \frac{x}{1. + |x|}$.
  \item \textit{sigmoid},return $f(x) = \frac{1.}{1. + exp(-x)}$.
  \item \textit{hard\_sigmoid}, hard sigmoid activation function.
  \item \textit{linear}, i.e. identity.
  \item \textit{softmax}, softmax activation function, return $f(x) = \frac{exp(x_i)}{\sum_i{exp(x_i)}}$
\end{itemize}

%%%%% Initializer Functions  %%%%%%%
\paragraph{Initializer Functions}
\label{initializersDNN}
Initializations define the way to set the initial random weights of TensorFlow-Keras layers. The keyword
arguments used to passing initializers to layers will depend on the layer. Usually it is simply
\xmlNode{kernel\_initializer} and \xmlNode{bias\_initializer}.
Available initializers include:
\begin{itemize}
  \item \textit{Zeros}, generates tensors initialized to 0.
  \item \textit{Ones}, generates tensors initialized to 1.
  \item \textit{Constant}, generates tensors initialized to a constant value.
  \item \textit{RandomNormal}, generates tensors with a normal distribution.
  \item \textit{RandomUniform}, generates tensors with a uniform distribution.
  \item \textit{TruncatedNormal}, generates a truncated normal distribution.
  \item \textit{VarianceScaling}, initializer capable of adapting its scale to the shape of weights.
  \item \textit{Orthogonal}, generates a random orthogonal matrix.
  \item \textit{Identity}, generates the identity matrix.
  \item \textit{lecun\_uniform}, LeCun uniform initializer.
    It draws samples from a uniform distribution within
    $[-limit, limit]$ where \textit{limit} is $sqrt(3/fanIn)$ where \textit{fanIn} is the number of input dimensions
    in the weight tensor.
  \item \textit{glorot\_normal}, Glorot normal initializer.
    It draws samples from a truncated normal distribution
    centered on 0 with $stddev = sqrt(2/(fanIn + fanOut))$ where \textit{fanIn} is the number of input dimensions
    in the weight tensor and \textit{fanOut} is the number of output dimensions in the weight tensors.
  \item \textit{glorot\_uniform}, Glorot uniform initializer.
    It draws samples from a uniform distribution within
    $[-limit, limit]$ where \textit{limit} is $sqrt(6/(fanIn+fanOut))$.
  \item \textit{he\_normal}, He normal initializer.
    It draws samples from a truncated normal distribution
    centered on 0 with $stddev = sqrt(2/fanIn)$.
  \item \textit{lecun\_normal}, LeCun normal initializer.
    It draws samples from a truncated normal distribution
    centered on 0 with $stddev = sqrt(1/fanIn)$.
  \item \textit{he\_uniform}, He uniform variance scaling initializer.
    It draws samples from a uniform distribution within
    $[-limit, limit]$ where \textit{limit} is $sqrt(6/fanIn)$ where \textit{fanIn} is the number of input dimensions
    in the weight tensor.
\end{itemize}

%%%%% Rgularizer Functions  %%%%%%%
\paragraph{Regularizer Functions}
\label{regularizersDNN}
Regularizers allow to apply penalities on layer parameters or layer activity during optimization.
These penalties are incorporated in the loss function that the network optimizes. The exact API
will depend on the layer, but the layers \xmlNode{Dense, Conv1D, Conv2D, and Conv3D} have a
unified API.
Available regularizers include:
\begin{itemize}
  \item \textit{l1}, l1 regularization
  \item \textit{l2}, l2 regularization
  \item \textit{l1\_l2}, l1 and l2 regularization
\end{itemize}

%%%%% Constraint Functions  %%%%%%%
\paragraph{Constraint Functions}
\label{constraintsDNN}
Functions from the \textit{constraint} module allow setting constraints on network parameters during optimization.
Available constraints include:
\begin{itemize}
  \item \textit{MaxNorm}, constrains the weights incident to each hidden unit to have a norm less than or equal to
    a desired value.
  \item \textit{NonNeg}, constrains the weights to be non-negative
  \item \textit{UnitNorm}, constrains the weights incident to each hidden unit to have unit norm.
  \item \textit{MinMaxNorm}, constrains the weights incident to each hidden unit to have the norm between a lower bound
    and an upper bound.
\end{itemize}

%%%%% ROM Model - KerasMLPClassifier  %%%%%%%
\paragraph{KerasMLPClassifier}
\label{KerasMLPClassifier}

Multi-Layer Perceptron (MLP) (or Artificial Neural Network - ANN), a class of feedforward
ANN, can be viewed as a logistic regression classifier where input is first transformed
using a non-linear transformation. This transformation probjects the input data into a
space where it becomes linearly separable. This intermediate layer is referred to as a
\textbf{hidden layer}. An MLP consists of at least three layers of nodes. Except for the
input nodes, each node is a neuron that uses a nonlinear \textbf{activation function}. MLP
utilizes a suppervised learning technique called \textbf{Backpropagation} for training.
Generally, a single hidden layer is sufficient to make MLPs a universal approximator.
However, many hidden layers, i.e. deep learning, can be used to model more complex nonlinear
relationships. The extra layers enable composition of features from lower layers, potentially
modeling complex data with fewer units than a similarly performing shallow network.

\zNormalizationPerformed{KerasMLPClassifier}

In order to use this ROM, the \xmlNode{ROM} attribute \xmlAttr{subType} needs to
be \xmlString{KerasMLPClassifier} (see the example below). This model can be initialized with
the following layers:

\begin{itemize}
  \DenseLayer
  \DropoutLayer
\end{itemize}

\textbf{Example:}
\begin{lstlisting}[style=XML,morekeywords={name,subType}]
<Simulation>
  ...
  <Models>
    ...
    <ROM name='aUserDefinedName' subType='KerasMLPClassifier'>
      <Features>X,Y</Features>
      <Target>Z</Target>
      <loss>mean_squared_error</loss>
      <metrics>accuracy</metrics>
      <batch_size>4</batch_size>
      <epochs>4</epochs>
      <optimizerSetting>
        <beta_1>0.9</beta_1>
        <optimizer>Adam</optimizer>
        <beta_2>0.999</beta_2>
        <epsilon>1e-8</epsilon>
        <decay>0.0</decay>
        <lr>0.001</lr>
      </optimizerSetting>
      <Dense name="layer1">
          <activation>relu</activation>
          <dim_out>15</dim_out>
      </Dense>
      <Dropout name="dropout1">
          <rate>0.2</rate>
      </Dropout>
      <Dense name="layer2">
          <activation>tanh</activation>
          <dim_out>8</dim_out>
      </Dense>
      <Dropout name="dropout2">
          <rate>0.2</rate>
      </Dropout>
      <Dense name="outLayer">
          <activation>sigmoid</activation>
      </Dense>
      <layer_layout>layer1, dropout1, layer2, dropout2, outLayer</layer_layout>
    </ROM>
    ...
  </Models>
  ...
</Simulation>
\end{lstlisting}

%%%%% ROM Model - KerasConvNetClassifier  %%%%%%%
\paragraph{KerasConvNetClassifier}
\label{KerasClassifier}

Convolutional Neural Network (CNN) is a deep learning algorithm which can take in an input image, assign
importance to various objects in the image and be able to differentiate one from the other. The
architecture of a CNN is analogous to that of the connectivity pattern of Neurons in the Human Brain
and was inspired by the organization of the Visual Cortex. Individual neurons respond to stimuli only
in a restricted region of the visual field known as the Receptive Field. A collection of such fields
overlap to cover the entire visual area. CNN is able to successfully capture the spatial and temporal
dependencies in an image through the applicaiton of relevant filters. The architecture performs
a better fitting to the image dataset due to the reduction in the number of parameters involved
and reusability of weights. In other words, the network can be trained to understand the sophistication
of the image better.

\zNormalizationPerformed{KerasConvNetClassifier}

In order to use this ROM, the \xmlNode{ROM} attribute \xmlAttr{subType} needs to
be \xmlString{KerasConvNetClassifier} (see the example below). This model can be initialized with
the following layers:

\begin{itemize}
  \DenseLayer
  \DropoutLayer
  \item \xmlNode{Conv1D}, \xmlDesc{optional field},
    \layerNameAttr
    In addition, this node also accepts the following subnodes
    \begin{itemize}
        \activation
        \dimOut
        \useBias
        \kernelSize
        \strides
        \padding
        \dataFormat
        \dilationRate
        \kernelInitializer
        \biasInitializer
        \kernelRegularizer
        \biasRegularizer
        \activityRegularizer
        \kernelConstraint
        \biasConstraint
    \end{itemize}

  \item \xmlNode{Conv2D}, \xmlDesc{optional field},
    In addition, this node also accepts the following subnodes
    \begin{itemize}
        \activation
        \dimOut
        \useBias
        \kernelSize
        \strides
        \padding
        \dataFormat
        \dilationRate
        \kernelInitializer
        \biasInitializer
        \kernelRegularizer
        \biasRegularizer
        \activityRegularizer
        \kernelConstraint
        \biasConstraint
    \end{itemize}

  \item \xmlNode{Conv3D}, \xmlDesc{optional field},
    In addition, this node also accepts the following subnodes
    \begin{itemize}
        \activation
        \dimOut
        \useBias
        \kernelSize
        \strides
        \padding
        \dataFormat
        \dilationRate
        \kernelInitializer
        \biasInitializer
        \kernelRegularizer
        \biasRegularizer
        \activityRegularizer
        \kernelConstraint
        \biasConstraint
    \end{itemize}

  \item \xmlNode{Flatten}, \xmlDesc{optional field},
    In addition, this node also accepts the following subnodes
    \begin{itemize}
        \dataFormat
    \end{itemize}

  \PoolingLayer{MaxPooling1D}
  \PoolingLayer{MaxPooling2D}
  \PoolingLayer{MaxPooling3D}
  \PoolingLayer{AveragePooling1D}
  \PoolingLayer{AveragePooling2D}
  \PoolingLayer{AveragePooling3D}
  \GlobalPoolingLayer{GlobalMaxPooling1D}
  \GlobalPoolingLayer{GlobalMaxPooling2D}
  \GlobalPoolingLayer{GlobalMaxPooling3D}
  \GlobalPoolingLayer{GlobalAveragePooling1D}
  \GlobalPoolingLayer{GlobalAveragePooling2D}
  \GlobalPoolingLayer{GlobalAveragePooling3D}
\end{itemize}

\textbf{Example:}
\begin{lstlisting}[style=XML,morekeywords={name,subType}]
<Simulation>
  ...
  <Models>
    ...
    <ROM name='aUserDefinedName' subType='KerasConvNetClassifier'>
      <Features>x1,x2</Features>
      <Target>labels</Target>
      <loss>mean_squared_error</loss>
      <metrics>accuracy</metrics>
      <batch_size>1</batch_size>
      <epochs>2</epochs>
      <plot_model>True</plot_model>
      <validation_split>0.25</validation_split>
      <num_classes>1</num_classes>
      <optimizerSetting>
        <beta_1>0.9</beta_1>
        <optimizer>Adam</optimizer>
        <beta_2>0.999</beta_2>
        <epsilon>1e-8</epsilon>
        <decay>0.0</decay>
        <lr>0.001</lr>
      </optimizerSetting>
      <Conv1D name="firstConv1D">
          <activation>relu</activation>
          <strides>1</strides>
          <kernel_size>2</kernel_size>
          <padding>valid</padding>
          <dim_out>32</dim_out>
      </Conv1D>
      <MaxPooling1D name="pooling1">
          <strides>2</strides>
          <pool_size>2</pool_size>
      </MaxPooling1D>
      <Conv1D name="SecondConv1D">
          <activation>relu</activation>
          <strides>1</strides>
          <kernel_size>2</kernel_size>
          <padding>valid</padding>
          <dim_out>32</dim_out>
      </Conv1D>
      <MaxPooling1D name="pooling2">
          <strides>2</strides>
          <pool_size>2</pool_size>
      </MaxPooling1D>
      <Flatten name="flatten">
      </Flatten>
      <Dense name="dense1">
          <activation>relu</activation>
          <dim_out>10</dim_out>
      </Dense>
      <Dropout name="dropout1">
          <rate>0.25</rate>
      </Dropout>
      <Dropout name="dropout2">
          <rate>0.25</rate>
      </Dropout>
      <Dense name="dense2">
          <activation>softmax</activation>
      </Dense>
      <layer_layout>firstConv1D, pooling1, dropout1, SecondConv1D, pooling2, dropout2, flatten, dense1, dense2</layer_layout>
    </ROM>
    ...
  </Models>
  ...
</Simulation>
\end{lstlisting}

%%%%% ROM Model - KerasLSTMClassifier  %%%%%%%
\paragraph{KerasLSTMClassifier}
\label{KerasClassifier}

Long Short Term Memory networks (LSTM) are a special kind of recurrent neural network, capable
of learning long-term dependencies. They work tremendously well on a large variety of problems, and
are now widely used. LSTMs are explicity designed to avoid the long-term dependency problem. Remembering
information for long periods of time is practically their default behavior, not something that they
struggle to learn.

\zNormalizationPerformed{KerasLSTMClassifier}

In order to use this ROM, the \xmlNode{ROM} attribute \xmlAttr{subType} needs to
be \xmlString{KerasLSTMClassifier} (see the example below). This model can be initialized with
the following layers:

\begin{itemize}
  \DenseLayer
  \DropoutLayer
  \item \xmlNode{LSTM}, \xmlDesc{required field}, long short-term memory layer.
    \layerNameAttr
    In addition, this node also accepts the following subnodes
    \begin{itemize}
        \activation
        \dimOut
        \recurrentActivation
        \dropout
        \recurrentDropout
        \returnSequence
        \useBias
        \kernelInitializer
        \recurrentInitializer
        \biasInitializer
        \unitForgetBias
        \kernelRegularizer
        \recurrentRegularizer
        \biasRegularizer
        \activityRegularizer
        \kernelConstraint
        \recurrentConstraint
        \biasConstraint
        \implementation
        \returnState
        \goBackwards
        \stateful
        \unroll
    \end{itemize}
\end{itemize}

\textbf{Example:}
\begin{lstlisting}[style=XML,morekeywords={name,subType}]
<Simulation>
  ...
  <Models>
    ...
    <ROM name='aUserDefinedName' subType='KerasLSTMClassifier'>
      <Features>x</Features>
      <Target>y</Target>
      <loss>categorical_crossentropy</loss>
      <metrics>accuracy</metrics>
      <batch_size>1</batch_size>
      <epochs>10</epochs>
      <validation_split>0.25</validation_split>
      <num_classes>26</num_classes>
      <optimizerSetting>
        <beta_1>0.9</beta_1>
        <optimizer>Adam</optimizer>
        <beta_2>0.999</beta_2>
        <epsilon>1e-8</epsilon>
        <decay>0.0</decay>
        <lr>0.001</lr>
      </optimizerSetting>
      <LSTM name="lstm1">
          <activation>tanh</activation>
          <dim_out>32</dim_out>
      </LSTM>
      <LSTM name="lstm2">
          <activation>tanh</activation>
          <dim_out>16</dim_out>
      </LSTM>
      <Dropout name="dropout">
          <rate>0.25</rate>
      </Dropout>
      <Dense name="dense">
          <activation>softmax</activation>
      </Dense>
      <layer_layout>lstm1,lstm2,dropout,dense</layer_layout>
    </ROM>
    ...
  </Models>
  ...
</Simulation>
\end{lstlisting}
