\subsubsection{HistorySetSync}
\label{HistorySetSync}

This PostProcessor performs the conversion from HistorySet to HistorySet
The conversion is made so that all histories are synchronized in time.
It can be used to allow the histories to be sampled at the same time instant.

There are two possible synchronization methods, specified through the \xmlNode{syncMethod} node.  If the
\xmlNode{syncMethod} is \xmlString{grid}, a \xmlNode{numberOfSamples} node is specified,
which yields an equally-spaced grid of time points. The output values for these points will be linearly derived
using nearest sampled time points, and the new HistorySet will contain only the new grid points.

The other methods are used by specifying \xmlNode{syncMethod} as \xmlString{all}, \xmlString{min}, or
\xmlString{max}.  For \xmlString{all}, the PostProcessor will iterate through the
existing histories, collect all the time points used in any of them, and use these as the new grid on which to
establish histories, retaining all the exact original values and interpolating linearly where necessary.
In the event of \xmlString{min} or \xmlString{max}, the PostProcessor will find the smallest or largest time
history, respectively, and use those time values as nodes to interpolate between.

\ppType{HistorySetSync}{HistorySetSync}

In the \xmlNode{PostProcessor} input block, the following XML sub-nodes are required,
independent of the \xmlAttr{subType} specified:

\begin{itemize}
   \item \xmlNode{pivotParameter}, \xmlDesc{string, required field}, ID of the temporal variable
   \item \xmlNode{extension}, \xmlDesc{string, required field}, type of extension when the sync process goes outside the boundaries of the history (zeroed or extended)
   \item \xmlNode{syncMethod}, \xmlDesc{string, required field}, synchronization strategy to employ (see
     description above).  Options are \xmlString{grid}, \xmlString{all}, \xmlString{max}, \xmlString{min}.
   \item \xmlNode{numberOfSamples}, \xmlDesc{integer, optional field}, required if \xmlNode{syncMethod} is
     \xmlString{grid}, number of new time samples
\end{itemize}
