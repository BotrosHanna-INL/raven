\subsubsection{HStoPSOperator}
\label{HStoPSOperator}

This PostProcessor performs the conversion from HistorySet to PointSet performing a projection of the output space.
\ppType{HStoPSOperator}{HStoPSOperator}
In the \xmlNode{PostProcessor} input block, the following XML sub-nodes are available:

\begin{itemize}
   \item \xmlNode{pivotParameter}, \xmlDesc{string, optional field}, ID of the temporal variable. Default is ``time''.
   \nb Used just in case the  \xmlNode{pivotValue}-based operation  is requested
    \item \xmlNode{operator}, \xmlDesc{string, optional field}, the operation to perform on the output space:
      \begin{itemize}
        \item \textbf{min}, compute the minimum of each variable along each single history
         \item \textbf{max}, compute the maximum of each variable along each single history
         \item \textbf{average}, compute the average of each variable along each single history
         \item \textbf{all}, join together all of the each variable in
           the history, and make the pivotParameter a regular
           parameter.  Unlike the min and max operators, this keeps
           all the data, just organized differently. This operator
           does this by propagating the other input parameters for
           each item of the pivotParameter.
           Table~\ref{operator_all_switch_before} shows an example
           HistorySet with input parameter x, pivot parameter t, and
           output parameter b and then
           Table~\ref{operator_all_switch_after} shows the resulting
           PointSet with input parameters x and t, and output
           parameter b. Note that which parameters are input and which
           are output in the resulting PointSet depends on the
           DataObject specification.
       \end{itemize}
        \nb This node can be inputted only if \xmlNode{pivotValue} and \xmlNode{row} are not present
     \item \xmlNode{pivotValue}, \xmlDesc{float, optional field}, the value of the pivotParameter with respect to the other outputs need to be extracted.
       \nb This node can be inputted only if \xmlNode{operator} and \xmlNode{row} are not present
     \item \xmlNode{pivotStrategy}, \xmlDesc{string, optional field}, The strategy to use for the pivotValue:
       \begin{itemize}
        \item \textbf{nearest}, find the value that is the nearest with respect the \xmlNode{pivotValue}
        \item \textbf{floor}, find the value that is the nearest with respect to the \xmlNode{pivotValue} but less then the  \xmlNode{pivotValue}
        \item \textbf{celing}, find the value that is the nearest with respect to the \xmlNode{pivotValue} but greater then the  \xmlNode{pivotValue}
        \item \textbf{interpolate}, if the exact  \xmlNode{pivotValue}  can not be found, interpolate using a linear approach
       \end{itemize}

       \nb Valid just in case \xmlNode{pivotValue} is present
     \item \xmlNode{row}, \xmlDesc{int, optional field}, the row index at which the outputs need to be extracted.
       \nb This node can be inputted only if \xmlNode{operator} and \xmlNode{pivotValue} are not present
\end{itemize}

This example will show how the XML input block would look like:

\begin{lstlisting}[style=XML,morekeywords={subType,debug,name,class,type}]
<Simulation>
  ...
  <Models>
    ...
    <PostProcessor name="HStoPSperatorRows" subType="HStoPSOperator">
      <row>-1</row>
    </PostProcessor>
    <PostProcessor name="HStoPSoperatorPivotValues" subType="HStoPSOperator">
        <pivotParameter>time</pivotParameter>
        <pivotValue>0.3</pivotValue>
    </PostProcessor>
    <PostProcessor name="HStoPSoperatorOperatorMax" subType="HStoPSOperator">
        <pivotParameter>time</pivotParameter>
        <operator>max</operator>
    </PostProcessor>
    <PostProcessor name="HStoPSoperatorOperatorMin" subType="HStoPSOperator">
        <pivotParameter>time</pivotParameter>
        <operator>min</operator>
    </PostProcessor>
    <PostProcessor name="HStoPSoperatorOperatorAverage" subType="HStoPSOperator">
        <pivotParameter>time</pivotParameter>
        <operator>average</operator>
    </PostProcessor>
    ...
  </Models>
  ...
</Simulation>
\end{lstlisting}

\begin{table}[!hbtp]
  \caption{Starting HistorySet for operator all}
  \label{operator_all_switch_before}
\begin{tabular}{l|l|l}
  x & t & b \\
  \hline
  5.0 &  &  \\
  \hline
  & 1.0 & 6.0 \\
  \hline
  & 2.0 & 7.0 \\
\end{tabular}
\end{table}

\begin{table}[!hbtp]
  \caption{Resulting PointSet after operator all}
  \label{operator_all_switch_after}
\begin{tabular}{l|l|l}
  x & t & b \\
  \hline
  5.0 & 1.0 & 6.0  \\
  \hline
  5.0 & 2.0 & 7.0 \\
\end{tabular}
\end{table}
