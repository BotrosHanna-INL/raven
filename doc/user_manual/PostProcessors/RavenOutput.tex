\subsubsection{RavenOutput}
\label{RavenOutput}
The \textbf{RavenOutput} post-processor is specifically used
to gather data from RAVEN output files and generate a PointSet suitable for plotting or other analysis.
It can do this in two modes: static and dynamic.  In static mode, the
PostProcessor reads from from several static XML output files produced by RAVEN.  In dynamic mode, the PostProcessor
reads from a single dynamic XML output file and builds a PointSet where the pivot parameter (e.g. time) is the
input and the requested values are returned for each of the pivot parameter values (e.g. points in time).  The
name for the pivot parameter will be taken directly from the XML structure.
%
Note: by default the PostProcessor operates in static mode; to read a dynamic file, the \xmlNode{dynamic} node must
be specified.
%
\ppType{RavenOutput}{RavenOutput}
%
\begin{itemize}
  \item \xmlNode{dynamic}, \xmlDesc{string, optional field}, if included will trigger reading a single dynamic
  file instead of multiple static files, unless the text of this field is \xmlString{false}, in which case it
  will return to the default (multiple static files).  \default(False)
  \item \xmlNode{File}, \xmlDesc{XML Node, required field}
  %
  For each file to be read by this postprocessor, an entry in the \xmlNode{Files} node must be added, and a
  \xmlNode{File} node must be added to the postprocessor input block.  The \xmlNode{File} requires two
  identifying attributes:
  \begin{itemize}
    \item \xmlAttr{name}, \xmlDesc{string, required field}, the RAVEN-assigned name of the file,
    \item \xmlAttr{ID}, \xmlDesc{float, optional field}, the floating point ID that will be unique to this
      file.  This will appear as an entry in the output \xmlNode{DataObject} and the corresponding column are
      the values extracted from this file.  If not specified, RAVEN will attempt to find a suitable integer ID
      to use, and a warning will be raised.

      When defining the \xmlNode{DataObject} that this postprocessor will write to, and when using the static
      (non-\xmlNode{dynamic}) form of the postprocessor, the \xmlNode{input} space should be given as
      \xmlString{ID}, and the output variables should be the outputs specified in the postprocessor. See the
      examples below.  In the data object, the variable values will be keyed on the \xmlString{ID} parameter.
  \end{itemize}
  Each value that needs to be extracted from the file needs to be specified by one of the following
  \xmlNode{output} nodes within the \xmlNode{File} node:
  \begin{itemize}
    \item \xmlNode{output}, \xmlDesc{|-separated string, required field},
           the specification of the output to extract from the file.
           RAVEN uses \texttt{xpath} as implemented in Python's \texttt{xml.etree} module to specify locations
           in XML.  For example, to search tags, use a path
           separated by forward slash characters (``/''), starting under the root; this means the root node should not
           be included in the path. See the example.  For more details on xpath options available, see
           \url{https://docs.python.org/2/library/xml.etree.elementtree.html#xpath-support}.
           %
           The \xmlNode{output} node requires the following attribute:
      \begin{itemize}
        \item \xmlAttr{name}, \xmlDesc{string, required field}, specifies the entry in the Data Object that
          this value should be stored under.
      \end{itemize}

  \end{itemize}
  %
\end{itemize}
\textbf{Example (Static):}
Using an example, let us have two input files, named \emph{in1.xml} and \emph{in2.xml}.  They appear as
follows.  Note that the name of the variables we want changes slightly between the XML; this is fine.

\textbf{\emph{in1.xml}}
\begin{lstlisting}[style=XML]
<BasicStatistics>
  <ans>
    <val1>6</val1>
    <val2>7</val2>
  </ans>
</BasicStatistics>
\end{lstlisting}
\textbf{\emph{in2.xml}}
\begin{lstlisting}[style=XML]
<ROM>
  <ans>
    <first>6.1</first>
    <second>7.1</second>
  </ans>
</BasicStatistics>
\end{lstlisting}

The RAVEN input to extract this information would appear as follows.
We include an example of defining the \xmlNode{DataObject} that this postprocessor will write out to, for
further clarity.

\begin{lstlisting}[style=XML]
<Simulation>
 ...
 <Files>
   <Input name='in1'>inp1.xml</Input>
   <Input name='in2'>inp2.xml</Input>
 </Files>
 ...
 <Models>
   ...
   <PostProcessor name='pp' subType='RavenOutput'>
     <File name='in1' ID='1'>
       <output name='first'>ans/val1</output>
       <output name='second'>ans/val2</output>
     </File>
     <File name='in2' ID='2'>
       <output name='first'>ans/first</output>
       <output name='second'>ans/second</output>
     </File>
   </PostProcessor>
   ...
 </Models>
 ...
 <DataObjects>
   ...
   <PointSet name='pointSetName'>
     <input>ID</input>
     <output>first,second</output>
   </PointSet>
   ...
 </DataObjects>
 ...
</Simulation>
\end{lstlisting}

\textbf{Example (Dynamic):}
For a dynamic example, consider this time-evolution of values example.  \emph{inFile.xml} is a RAVEN dynamic
XML output.

\textbf{\emph{in1.xml}}
\begin{lstlisting}[style=XML]
<BasicStatistics type='Dynamic'>
  <time value='0.0'>
    <ans>
      <val1>6</val1>
      <val2>7</val2>
    </ans>
  <\time>
  <time value='1.0'>
    <ans>
      <val1>9</val1>
      <val2>10</val2>
    </ans>
  <\time>
</BasicStatistics>
\end{lstlisting}
The RAVEN input to extract this information would appear as follows:
\begin{lstlisting}[style=XML]
<Simulation>
 ...
 <Files>
   <Input name='inFile'>inFile.xml</Input>
 </Files>
 ...
 <Models>
   ...
   <PostProcessor name='pp' subType='RavenOut'>
     <dynamic>true</dynamic>
     <File name='inFile'>
       <output name='first'>ans|val1</output>
     </File>
   </PostProcessor>
   ...
 </Models>
 ...
</Simulation>
\end{lstlisting}
The resulting PointSet has \emph{time} as an input and \emph{first} as an output.
