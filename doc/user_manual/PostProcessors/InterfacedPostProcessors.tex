\subsubsection{Interfaced}
\label{Interfaced}
The \textbf{Interfaced} post-processor is a Post-Processor that allows the user
to create its own Post-Processor. While the External Post-Processor (see
Section~\ref{External} allows the user to create case-dependent
Post-Processors, with this new class the user can create new general
purpose Post-Processors.
%

\ppType{Interfaced}{Interfaced}

\begin{itemize}
  \item \xmlNode{method}, \xmlDesc{comma separated string, required field},
  lists the method names of a method that will be computed (each
  returning a post-processing value). All available methods need to be included
  in the ``/raven/framework/PostProcessorFunctions/'' folder
\end{itemize}

\textbf{Example:}
\begin{lstlisting}[style=XML,morekeywords={subType,debug,name,class,type}]
<Simulation>
  ...
  <Models>
    ...
    <PostProcessor name="example" subType='InterfacedPostProcessor'verbosity='debug'>
       <method>testInterfacedPP</method>
       <!--Here, the xml nodes required by the chosen method have to be
       included.
        -->
    </PostProcessor>
    ...
  </Models>
  ...
</Simulation>
\end{lstlisting}

All the \textbf{Interfaced} post-processors need to be contained in the
``/raven/framework/PostProcessorFunctions/'' folder. In fact, once the
\textbf{Interfaced} post-processor is defined in the RAVEN input file, RAVEN
search that the method of the post-processor is located in such folder.

The class specified in the \textbf{Interfaced} post-processor has to inherit the
PostProcessorInterfaceBase class and the user must specify this set of
methods:
\begin{itemize}
  \item initialize: in this method, the internal parameters of the
  post-processor are initialized. Mandatory variables that needs to be
  specified are the following:
\begin{itemize}
  \item self.inputFormat: type of dataObject expected in input
  \item self.outputFormat: type of dataObject generated in output
\end{itemize}
  \item readMoreXML: this method is in charge of reading the PostProcessor xml
  node, parse it and fill the PostProcessor internal variables.
  \item run: this method performs the desired computation of the dataObject.
\end{itemize}

\begin{lstlisting}[language=python]
from PostProcessorInterfaceBaseClass import PostProcessorInterfaceBase
class testInterfacedPP(PostProcessorInterfaceBase):
  def initialize(self)
  def readMoreXML(self,xmlNode)
  def run(self,inputDic)
\end{lstlisting}

\paragraph{Data Format}
The user is not allowed to modify directly the DataObjects, however the
content of the DataObjects is available in the form of a python dictionary.
Both the dictionary give in input and the one generated in the output of the
PostProcessor are structured as follows:

\begin{lstlisting}[language=python]
inputDict = {'data':{}, 'metadata':{}}
\end{lstlisting}

where:

\begin{lstlisting}[language=python]
inputDict['data'] = {'input':{}, 'output':{}}
\end{lstlisting}

In the input dictonary, each input variable is listed as a dictionary that
contains a numpy array with its own values as shown below for a simplified
example

\begin{lstlisting}[language=python]
inputDict['data']['input'] = {'inputVar1': array([ 1.,2.,3.]),
                              'inputVar2': array([4.,5.,6.])}
\end{lstlisting}

Similarly, if the dataObject is a PointSet then the output dictionary is
structured as follows:

\begin{lstlisting}[language=python]
inputDict['data']['output'] = {'outputVar1': array([ .1,.2,.3]),
                               'outputVar2':array([.4,.5,.6])}
\end{lstlisting}

Howevers, if the dataObject is a HistorySet then the output dictionary is
structured as follows:

\begin{lstlisting}[language=python]
inputDict['data']['output'] = {'hist1': {}, 'hist2':{}}
\end{lstlisting}

where

\begin{lstlisting}[language=python]
inputDict['output']['data'][hist1] = {'time': array([ .1,.2,.3]),
                              'outputVar1':array([ .4,.5,.6])}
inputDict['output']['data'][hist2] = {'time': array([ .1,.2,.3]),
                              'outputVar1':array([ .14,.15,.16])}
\end{lstlisting}
