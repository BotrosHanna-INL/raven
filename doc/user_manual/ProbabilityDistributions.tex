\section{Distributions}
\label{sec:distributions}
\newcommand{\distname}[1]{\textbf{#1}}
\newcommand{\distattrib}[1]{\textit{#1}}

%%%%%%%%%%%%%%%%%%%%%%%%%%%%%%%%%%%%%%%%%%%%%%%%%%%%%%%%%%%%%%%%%%%%%%%%%%%%%%%%
% If you are confused by the input of this document, please make sure you see
% these defined commands first. There is no point writing the same thing over
% and over and over and over and over again, so these will help us reduce typos,
% by just editing a template sentence.
\newcommand{\nameDescription}{\xmlAttr{name},
  \xmlDesc{required string attribute}, user-defined name of this distribution.
  %
  \nb As with other objects, this identifier can be used to reference this
  specific entity from other input blocks in the XML.}
\newcommand{\specBlock}[2]{The specifications of this distribution must be
  defined within #1 \xmlNode{#2} XML block.}
\newcommand{\attrIntro}{This XML node accepts one attribute:}
\newcommand{\attrsIntro}{This XML node accepts the following attributes:}
\newcommand{\subnodeIntro}{This distribution can be initialized with the
  following child node:}
\newcommand{\subnodesIntro}{This distribution can be initialized with the
  following children nodes:}
%%%%%%%%%%%%%%%%%%%%%%%%%%%%%%%%%%%%%%%%%%%%%%%%%%%%%%%%%%%%%%%%%%%%%%%%%%%%%%%%

%\maljdan{Do we want to provide the equations of each distribution?}
%\alfoa{TBH I do not think so. It is not a theory manual... I put few equations in the ROM section just to explain better the meaning of some parameters.}

RAVEN provides support for several probability distributions.
%
Currently, the user can choose among several 1-dimensional distributions and
$N$-dimensional ones, either custom or multidimensional normal.

The user will specify the probability distributions, that need to be used during
the simulation, within the \xmlNode{Distributions} XML block:
\begin{lstlisting}[style=XML]
<Simulation>
   ...
  <Distributions>
    <!-- All the necessary distributions will be listed here -->
  </Distributions>
  ...
</Simulation>
\end{lstlisting}

In the next two sub-sections, the input requirements for all of the
distributions are reported.

%%%%%% 1-Dimensional Probability distributions
\subsection{1-Dimensional Probability Distributions}
\label{subsec:1dDist}
This sub-section is organized in two different parts: 1) continuous 1-D
distributions and 2) discrete 1-D distributions.
%
These two paragraphs cover all the requirements for using the different
distribution entities.
%
%%%%%% paragraph 1-Dimensional Continuous Distributions.
\subsubsection{1-Dimensional Continuous Distributions}
\label{subsubsec:1DContinuous}
In this paragraph all the 1-D distributions currently available in RAVEN are
reported.

Firstly, all the probability distributions functions in the code can be
truncated by using the following keywords:
\begin{lstlisting}[style=XML]
<Distributions>
   ...
   <aDistributionType>
      ...
      <lowerBound>aFloatValue</lowerBound>
      <upperBound>aFloatValue</upperBound>
      ...
   </aDistributionType>
</Distributions>
\end{lstlisting}
Each distribution has a pre-defined, default support (domain) based on its
definition, however these domains can be shifted/stretched using the appropriate
\xmlNode{low} and \xmlNode{high} parameters where applicable, and/or truncated
using the nodes in the example above, namely \xmlNode{lowerBound} and
\xmlNode{upperBound}.
For example, the Normal distribution domain is $[-\infty,+\infty]$, and thus
cannot be shifted or stretched, as it is already unbounded, but can be
truncated.
%
RAVEN currently provides support for 13 1-Dimensional distributions.
%
In the following paragraphs, all the input requirements are reported and
commented.

%%%%%% Beta
\paragraph{Beta Distribution}
\label{Beta}
The \distname{Beta} distribution is parameterized by two positive shape parameters, denoted by
$\alpha$ and $\beta$, that appear as exponents of the random variable. Its default support
(domain) is $x \in [0, 1]$.
%
The distribution domain can be changed, specifying new boundaries, to fit the user's needs.
%
The user can specify a \distname{Beta} distribution in two ways.  The standard
is to provide the parameters \xmlNode{low}, \xmlNode{high}, \xmlNode{alpha},
and \xmlNode{beta}.  Alternatively, to approximate a normal
distribution that falls to 0 at the endpoints, the user may provide
the parameters \xmlNode{low}, \xmlNode{high}, and \xmlNode{peakFactor}. The
peak factor is a value between 0 and 1 that determines the peakedness of the
distribution.  At 0 it is dome-like ($\alpha=\beta=4$) and at 1 it is very
strongly peaked around the mean ($\alpha=\beta=100$).  A reasonable approximation
to a Gaussian normal is a peak factor of 0.5.

\specBlock{a}{Beta}
%
\attrIntro
\vspace{-5mm}
\begin{itemize}
  \itemsep0em
  \item \nameDescription
\end{itemize}
\vspace{-5mm}
\subnodesIntro
\begin{itemize}
  \item Standard initialization:
    \begin{itemize}
    \item \xmlNode{alpha}, \xmlDesc{float, conditional required parameter}, first shape
     parameter.  If specified, \xmlNode{beta} must also be inputted and
     \xmlNode{peakFactor} can not be specified.
%
     \item \xmlNode{beta}, \xmlDesc{float, conditional required parameter}, second shape
      parameter.  If specified, \xmlNode{alpha} must also be inputted and
      \xmlNode{peakFactor} can not be specified.
%
       \item \xmlNode{low}, \xmlDesc{float, optional parameter}, lower domain
       boundary. \default{0.0}
 %
       \item \xmlNode{high}, \xmlDesc{float, optional parameter}, upper domain,
         boundary. \default{1.0}
      \end{itemize}
 %
     \item Alternative initialization:
     \begin{itemize}
        \item \xmlNode{peakFactor}, \xmlDesc{float, optional parameter}, alternative
         to specifying \xmlNode{alpha} and \xmlNode{beta}.  Acceptable values range from
        0 to 1.
%
       \item \xmlNode{low}, \xmlDesc{float, optional parameter}, lower domain
       boundary. \default{0.0}
 %
       \item \xmlNode{high}, \xmlDesc{float, optional parameter}, upper domain,
         boundary. \default{1.0}
      \end{itemize}
  %
\end{itemize}

Example:
\begin{lstlisting}[style=XML]
<Distributions>
  ...
  <Beta name='aUserDefinedName'>
     <low>aFloatValue</low>
     <high>aFloatValue</high>
     <alpha>aFloatValue</alpha>
     <beta>aFloatValue</beta>
  </Beta>
  <Beta name='aUserDefinedName2'>
     <low>aFloatValue</low>
     <high>aFloatValue</high>
     <peakFactor>aFloatValue</peakFactor>
  </Beta>
  ...
</Distributions>
\end{lstlisting}

%%%%%% Exponential
\paragraph{Exponential Distribution}
\label{Exponential}
The \distname{Exponential} distribution has a default support of
$x \in [0, +\infty)$.

\specBlock{an}{Exponential}
%
\attrIntro
\vspace{-5mm}
\begin{itemize}
  \itemsep0em
  \item \nameDescription
\end{itemize}
\vspace{-5mm}
\subnodeIntro
\begin{itemize}
  \item \xmlNode{lambda}, \xmlDesc{float, required parameter}, rate parameter.
  \item \xmlNode{low}, \xmlDesc{float, optional parameter}, lower domain
     boundary. \default{0.0}
  %
\end{itemize}

Example:
\begin{lstlisting}[style=XML]
<Distributions>
  ...
  <Exponential name='aUserDefinedName'>
    <lambda>aFloatValue</lambda>
    <low>aFloatValue</low>
  </Exponential>
  ...
</Distributions>
\end{lstlisting}

%%%%%% Gamma
\paragraph{Gamma Distribution}
\label{Gamma}
The \distname{Gamma} distribution is a two-parameter family of continuous
probability distributions.
%
The common exponential distribution and $\chi$-squared distribution are special
cases of the gamma distribution.
%
Its default support is $x \in [0,+\infty]$.

\specBlock{a}{Gamma}
%
\attrIntro
\vspace{-5mm}
\begin{itemize}
  \itemsep0em
  \item \nameDescription
\end{itemize}
\vspace{-5mm}
\subnodesIntro
\begin{itemize}
  \item \xmlNode{alpha}, \xmlDesc{float, required parameter}, shape parameter.
  %
  \item \xmlNode{beta}, \xmlDesc{float, optional parameter}, 1/scale or the
  inverse scale parameter. \default{1.0}
  \item \xmlNode{low}, \xmlDesc{float, optional parameter}, lower domain
  boundary. \default{0.0}
  %
\end{itemize}

Example:
\begin{lstlisting}[style=XML]
<Distributions>
  ...
  <Gamma name='aUserDefinedName'>
    <alpha>aFloatValue</alpha>
    <beta>aFloatValue</beta>
    <low>aFloatValue</low>
  </Gamma>
  ...
</Distributions>
\end{lstlisting}

%%%%%% Laplace
\paragraph{Laplace Distribution}
\label{Laplace}
The \distname{Laplace} distribution is a two-parameter continuous
probability distribution.  It is the distribution of the differences
between two independent random variables with identical exponential
distributions.
%
Its default support is $x \in (-\infty,+\infty)$.

\specBlock{a}{Laplace}
%
\attrIntro
\vspace{-5mm}
\begin{itemize}
  \itemsep0em
  \item \nameDescription
\end{itemize}
\vspace{-5mm}
\subnodesIntro
\begin{itemize}
\item \xmlNode{location}, \xmlDesc{float, required parameter},
  determines the location or shift of the distribution.
\item \xmlNode{scale}, \xmlDesc{float, required parameter}, must be
  greater than 0, and determines how spread out the distribution is.
\end{itemize}

Example:
\begin{lstlisting}[style=XML]
<Distributions>
  ...
  <Laplace name='aUserDefinedName'>
    <location>aFloatValue</location>
    <scale>aFloatValue</scale>
  </Laplace>
  ...
</Distributions>
\end{lstlisting}

%%%%%% Logistic
\paragraph{Logistic Distribution}
\label{Logistic}
The \distname{Logistic} distribution is similar to the
normal distribution with a CDF that is an instance of a logistic function ($Cdf(x) = \frac{1}{1+e^{-\frac{(x-location)}{scale})}}$).
%
It resembles the normal distribution in shape but has heavier tails (higher
kurtosis).
%
Its default support is $x \in [-\infty,+\infty]$.

\specBlock{a}{Logistic}
%
\attrIntro
\vspace{-5mm}
\begin{itemize}
  \itemsep0em
  \item \nameDescription
\end{itemize}
\vspace{-5mm}
\subnodesIntro
\begin{itemize}
  \item \xmlNode{location}, \xmlDesc{float, required parameter}, the
  distribution
  mean.
  %
  \item \xmlNode{scale}, \xmlDesc{float, required parameter}, scale parameter
  that
  is proportional to the standard deviation ($\sigma ^{2}=\frac{1}{3}\pi^{2}scale^{2} $).
  %
\end{itemize}

Example:
\begin{lstlisting}[style=XML]
<Distributions>
  ...
  <Logistic name='aUserDefinedName'>
    <location>aFloatValue</location>
    <scale>aFloatValue</scale>
  </Logistic>
  ...
</Distributions>
\end{lstlisting}

%%%%%% LogNormal
\paragraph{LogNormal Distribution}
\label{LogNormal}
The \distname{LogNormal} distribution is a distribution with the
logarithm of the random variable being normally distributed.
%
Its default support is $x \in [0, +\infty]$.

\specBlock{a}{LogNormal}
%
\attrIntro
\vspace{-5mm}
\begin{itemize}
  \itemsep0em
  \item \nameDescription
\end{itemize}
\vspace{-5mm}
\subnodesIntro
\begin{itemize}
  \item \xmlNode{mean}, \xmlDesc{float, required parameter}, the log of the distribution
  mean or expected value.
  %
  \item \xmlNode{sigma}, \xmlDesc{float, required parameter}, standard
  deviation.
  \item \xmlNode{low}, \xmlDesc{float, optional parameter}, lower domain
  boundary. \default{0.0}
  %
\end{itemize}
\nb The \xmlNode{mean} and \xmlNode{sigma} listed above are NOT the mean and standard deviation of the
distribution; they are the mean and standard deviation of the log of the distribution.  Using the following
notation:
\begin{itemize}
  \item $\mu_\ell$: the $\mu$ parameter of the lognormal distribution, which RAVEN expects in the
    \xmlNode{mean} node;
  \item $\sigma_\ell$: the $\sigma$ parameter of the lognormal distribution, which RAVEN expects in the
    \xmlNode{sigma} node;
  \item $M$: the user-desired mean value of the distribution;
  \item $S$: the user-desired standard deviation of the distribution;
\end{itemize}
a conversion is defined to translate from mean $M$ and standard deviation $S$ into the parameters RAVEN
expects:
\begin{equation}
  \mu_\ell = \log\left(\frac{M}{\sqrt{1+\frac{S^2}{M^2}}}\right),
\end{equation}
\begin{equation}
  \sigma_\ell = \sqrt{\log{1+\frac{S^2}{M^2} }}.
\end{equation}

Example:
\begin{lstlisting}[style=XML]
<Distributions>
  ...
  <LogNormal name='aUserDefinedName'>
    <mean>aFloatValue</mean>
    <sigma>aFloatValue</sigma>
    <low>aFloatValue</low>
  </LogNormal>
  ...
</Distributions>
\end{lstlisting}

%%%%%% LogUniform
\paragraph{LogUniform Distribution}
\label{LogNormal}
The \distname{LogNormal} distribution is a distribution associated to
a variable $y=h(x)=e^{x}$ where variable x is uniform distributed.
This distribution supports not only the case  $y=h(x)=e^{x}$ (natural case) but also
the case where $y=h(x)=10^{x}$ (decimal case).
%

Its default support is $x \in [h(lowerBound),h(upperBound)]$.

\specBlock{a}{LogUniform}
%
\attrIntro
\vspace{-5mm}
\begin{itemize}
  \itemsep0em
  \item \nameDescription
\end{itemize}
\vspace{-5mm}
\subnodesIntro
\begin{itemize}
  \item \xmlNode{lowerBound}, \xmlDesc{float, required parameter}, domain lower boundary.
  %
  \item \xmlNode{upperBound}, \xmlDesc{float, required parameter}, domain upper boundary.
  %
  \item \xmlNode{base}, \xmlDesc{string, required parameter}, case type (decimal or natural).
  %
\end{itemize}

Example:
\begin{lstlisting}[style=XML]
<Distributions>
  ...
  <LogUniform name="x_dist">
    <upperBound>1.0</upperBound>
    <lowerBound>3.0</lowerBound>
    <base>natural</base>
  </LogUniform>
  ...
</Distributions>
\end{lstlisting}

%%%%%% Normal
\paragraph{Normal Distribution}
\label{Normal}
The \distname{Normal} distribution is an extremely
useful continuous distribution.
%
Its utility is due to the central limit theorem, which states that, under mild
conditions, the mean of many random variables independently drawn from the same
distribution is distributed approximately normally, irrespective of the form of
the original distribution.
%
Its default support is $x \in [-\infty, +\infty]$.

\specBlock{a}{Normal}
%
\attrIntro
\vspace{-5mm}
\begin{itemize}
  \itemsep0em
  \item \nameDescription
\end{itemize}
\vspace{-5mm}
\subnodesIntro
\begin{itemize}
  \item \xmlNode{mean}, \xmlDesc{float, required parameter}, the distribution
  mean
  or expected value.
  %
  \item \xmlNode{sigma}, \xmlDesc{float, required parameter}, the standard
  deviation.
  %
\end{itemize}

Example:
\begin{lstlisting}[style=XML]
<Distributions>
  ...
  <Normal name='aUserDefinedName'>
    <mean>aFloatValue</mean>
    <sigma>aFloatValue</sigma>
  </Normal>
  ...
</Distributions>
\end{lstlisting}

%%%%%% Triangular
\paragraph{Triangular Distribution}
\label{Triangular}
The \distname{Triangular} distribution is a continuous distribution that has a
triangular shape for its PDF.
%
%It is often used where the distribution is only vaguely known.
%
Like the uniform distribution, upper and lower limits are ``known,'' but a
``best guess,'' of the mode or center point is also added.
%
It has been recommended as a ``proxy'' for the beta distribution.
%
Its default support is $x \in [min,max]$.

\specBlock{a}{Triangular}
%
\attrIntro
\vspace{-5mm}
\begin{itemize}
  \itemsep0em
  \item \nameDescription
\end{itemize}
\vspace{-5mm}
\subnodesIntro
\begin{itemize}
  \item \xmlNode{apex}, \xmlDesc{float, required parameter}, peak location
  %``best guess'', also called, peak factor.
  %
  \item \xmlNode{min}, \xmlDesc{float, required parameter}, domain lower
  boundary.
  %
  \item \xmlNode{max}, \xmlDesc{float, required parameter}, domain upper
  boundary.
  %
\end{itemize}

Example:
\begin{lstlisting}[style=XML]
<Distributions>
  ...
  <Triangular name='aUserDefinedName'>
    <apex>aFloatValue</apex>
    <min>aFloatValue</min>
    <max>aFloatValue</max>
  </Triangular>
  ...
</Distributions>
\end{lstlisting}

%%%%%% Uniform
\paragraph{Uniform Distribution}
\label{Uniform}
The \distname{Uniform} distribution is a continuous distribution with a
rectangular-shaped PDF.
%
It is often used where the distribution is only vaguely known, but upper and
lower limits are known.
%
Its default support is $x \in [lower,upper]$.

\specBlock{a}{Uniform}
%
\attrIntro
\vspace{-5mm}
\begin{itemize}
  \itemsep0em
  \item \nameDescription
\end{itemize}
\vspace{-5mm}
\subnodesIntro
\begin{itemize}
  \item \xmlNode{lowerBound}, \xmlDesc{float, required parameter}, domain lower
  boundary.
  %
  \item \xmlNode{upperBound}, \xmlDesc{float, required parameter}, domain upper
  boundary.
  %
\end{itemize}
\nb Since the Uniform distribution is a rectangular-shaped PDF, the truncation does not have any effect;
 this is the reason why the children nodes are the ones generally used for truncated distributions.
Example:
\begin{lstlisting}[style=XML]
<Distributions>
  ...
  <Uniform name='aUserDefinedName'>
    <lowerBound>aFloatValue</lowerBound>
    <upperBound>aFloatValue</upperBound>
  </Uniform>
  ...
</Distributions>
\end{lstlisting}

%%%%%% Weibull
\paragraph{Weibull Distribution}
\label{Weibull}
The \distname{Weibull} distribution is a continuous distribution that is often
used in the field of failure analysis; in particular, it can mimic distributions
where the failure rate varies over time.
%
If the failure rate is:
\vspace{-5mm}
\begin{itemize}
  \itemsep0em
  \item constant over time, then $k = 1$, suggests that items are failing from
  random events;
  \item decreases over time, then $k < 1$, suggesting ``infant mortality'';
  \item increases over time, then $k > 1$, suggesting ``wear out'' - more likely
  to fail as time goes by.
  %
\end{itemize}
\vspace{-5mm}
Its default support is $x \in [0, +\infty)$.

\specBlock{a}{Weibull}
%
\attrIntro
\vspace{-5mm}
\begin{itemize}
  \itemsep0em
  \item \nameDescription
\end{itemize}
\vspace{-5mm}
\subnodesIntro
\begin{itemize}
  \item \xmlNode{k}, \xmlDesc{float, required parameter}, shape parameter.
  %
  \item \xmlNode{lambda}, \xmlDesc{float, required parameter}, scale parameter.
  \item \xmlNode{low}, \xmlDesc{float, optional parameter}, lower domain
  boundary. \default{0.0}
  %
\end{itemize}

Example:
\begin{lstlisting}[style=XML]
<Distributions>
  ...
  <Weibull name='aUserDefinedName'>
    <lambda>aFloatValue</lambda>
    <k>aFloatValue</k>
    <low>aFloatValue</low>
  </Weibull>
  ...
</Distributions>
\end{lstlisting}

%%%%%% Custom1D
\paragraph{Custom1D Distribution}
\label{Custom1D}
The \distname{Custom1D} distribution is a custom continuous distribution that can be initialized from a dataObject
generated by RAVEN.
This distribution cannot be initialized from a dataObject directly but through a .csv file.
This file must contain the values of either cdf or pdf of the random variable sampled along the range of the desired 
random variable.
In the distribution block of the RAVEN input file, the user needs to specify which file (including its working directory)
needs to be used to initialize the distribution. In addition, the user is required to specify which type (cdf or pdf) or values
are contained in the file and also the IDs of both the random variable and cdf/pdf.
Thus the csv file contains a set of points that samples the function $pdf(x)$ or $cdf(x)$ for several values of the stochastic variable $x$. 
The user needs to specify which variable IDs correspond to $x$ and $pdf(x)$ (or $cdf(x)$).
The distribution create a fourth order spline interpolation from the provided input points.
%
Note that the support of this distribution is set between the minimum and maximum values of the random variable which are 
specified in the distribution input file.

Refer to the test example ($tests/framework/test_distributionCustom1D.xml$) for more clarification.

\specBlock{a}{Custom1D}
%
\attrIntro
\vspace{-5mm}
\begin{itemize}
  \itemsep0em
  \item \nameDescription
\end{itemize}
\vspace{-5mm}
\subnodesIntro
\begin{itemize}
  \item \xmlNode{dataFilename}, \xmlDesc{string, required parameter}, file name to be used to initialize the distribution.
  \item \xmlNode{workingDir}, \xmlDesc{string, optional parameter}, relative working directory that contains the input file.
  \item \xmlNode{functionType}, \xmlDesc{string, required parameter}, type of initialization values specified in the input file (pdf or cdf).
  \item \xmlNode{variableID}, \xmlDesc{string, required parameter}, ID of the variable contained in the input file.
  \item \xmlNode{functionID}, \xmlDesc{string, required parameter}, ID of the function associated to the variableID contained in the input file.
\end{itemize}

Example:
\begin{lstlisting}[style=XML]
<Distributions>
  ...
    <Custom1D name="pdf_custom">
      <dataFilename>PointSetFile2_dump.csv</dataFilename>
      <functionID>pdf_values</functionID>
      <variableID>x</variableID>
      <functionType>pdf</functionType>
      <workingDir>custom1D/</workingDir>
    </Custom1D>
    <Custom1D name="cdf_custom">
      <dataFilename>PointSetFile3_dump.csv</dataFilename>
      <functionID>cdf_values</functionID>
      <variableID>x</variableID>
      <functionType>cdf</functionType>
      <workingDir>custom1D/</workingDir>
    </Custom1D>
  ...
</Distributions>
\end{lstlisting}


The example above initializes two distributions from two .csv files. 
For example, the first distribution retrieves the pdf values, located in the column with label $pdf_values$, for several locations of the variable located in the column 
with label $x$ in the file $PointSetFile2_dump.csv$.


%%%%%% paragraph 1-Dimensional Discrete Distributions.
\subsubsection{1-Dimensional Discrete Distributions.}
\label{subsubsec:1DDiscrete}
RAVEN currently supports 3 discrete distributions.
%
In the following paragraphs, the input requirements are reported.

%%%%%% Bernoulli
\paragraph{Bernoulli Distribution}
\label{Bernoulli}
The \distname{Bernoulli} distribution is a discrete distribution of the outcome
of a single trial with only two results, 0 (failure) or 1 (success), with a
probability of success \distattrib{p}.
%
It is the simplest building block on which other discrete distributions of
sequences of independent Bernoulli trials can be based.
%
Basically, it is the binomial distribution (k = 1, \distattrib{p}) with only
one trial.
%
Its default support is $k \in {0, 1}$.

\specBlock{a}{Bernoulli}
%
\attrIntro
\vspace{-5mm}
\begin{itemize}
  \itemsep0em
  \item \nameDescription
\end{itemize}
\vspace{-5mm}
\subnodeIntro
\begin{itemize}
  \item \xmlNode{p}, \xmlDesc{float, required parameter}, probability of
  success.
  %
 \end{itemize}
Example:
\begin{lstlisting}[style=XML]
<Distributions>
  ...
  <Bernoulli name='aUserDefinedName'>
    <p>aFloatValue</p>
  </Bernoulli>
  ...
</Distributions>
\end{lstlisting}

%%%%%% Binomial
\paragraph{Binomial Distribution}
\label{Binomial}
The \distname{Binomial} distribution is the discrete probability distribution of
the number of successes in a sequence of \distattrib{n} independent yes/no
experiments, each of which yields success with probability \distattrib{p}.
%
Its default support is $k \in {0, 1, 2, ..., n}$.

\specBlock{a}{Binomial}
%
\attrIntro
\vspace{-5mm}
\begin{itemize}
  \itemsep0em
  \item \nameDescription
\end{itemize}
\vspace{-5mm}
\subnodesIntro
\begin{itemize}
  \item \xmlNode{p}, \xmlDesc{float, required parameter}, probability of
  success.
  %
  \item \xmlNode{n}, \xmlDesc{integer, required parameter}, number of
  experiments.
  %
\end{itemize}

Example:
\begin{lstlisting}[style=XML]
<Distributions>
  ...
  <Binomial name='aUserDefinedName'>
    <n>aIntegerValue</n>
    <p>aFloatValue</p>
  </Binomial>
  ...
</Distributions>
\end{lstlisting}

%%%%%% Geometric
\paragraph{Geometric Distribution}
\label{Geometric}
The \distname{Geometric} distribution is a one-parameter discrete probability distribution.
%
The distribution uses the probability $p$ that trial will be
successful.  The geometric distribution gives the probability of
observing $k$ trials before the first success.
%
Its support is $k \in {0, 1, 2, ..., n}$.

\specBlock{a}{Geometric}

\attrIntro
\vspace{-5mm}
\begin{itemize}
  \itemsep0em
\item \nameDescription
\end{itemize}
\vspace{-5mm}
\subnodesIntro
\begin{itemize}
\item \xmlNode{p}, \xmlDesc{float, required parameter}, the success fraction for the trials.
\end{itemize}

Example:
\begin{lstlisting}[style=XML]
<Distributions>
  ...
  <Geometric name='aUserDefinedName'>
    <p>aFloatValue</p>
  </Geometric>
  ...
</Distributions>
\end{lstlisting}

%%%%%% Poisson
\paragraph{Poisson Distribution}
\label{Poisson}
The \distname{Poisson} distribution is a discrete probability distribution that
expresses the probability of a given number of events occurring in a fixed
interval of time and/or space if these events occur with a known average rate
and independently of the time since the last event.
%
Its default support is $k \in {1, 2, 3, 4, ...}$.

\specBlock{a}{Poisson}
%
\attrIntro
\vspace{-5mm}
\begin{itemize}
  \itemsep0em
  \item \nameDescription
\end{itemize}
\vspace{-5mm}
\subnodeIntro
\begin{itemize}
  \item \xmlNode{mu}, \xmlDesc{float, required parameter}, mean rate of
  events/time.
  %
\end{itemize}

Example:
\begin{lstlisting}[style=XML]
<Distributions>
  ...
  <Poisson name='aUserDefinedName'>
    <mu>aFloatValue</mu>
  </Poisson>
  ...
</Distributions>
\end{lstlisting}


%%%%%% Categorical
\paragraph{Categorical Distribution}
\label{Categorical}
The \distname{Categorical} distribution is a discrete distribution that describes the result of a random variable that can have $K$ possible outcomes. 
The probability of each outcome is separately specified.
The possible outcomes must be only numerical values (either integer or float numbers). No string can be assigned to any outcome.
%
There is not necessarily an underlying ordering of these outcomes, but labels are assigned in describing the distribution (in the range $1$ to $K$).
%
\specBlock{a}{Categorical}
%
\attrIntro
\vspace{-5mm}
\begin{itemize}
  \itemsep0em
  \item \nameDescription
\end{itemize}
\vspace{-5mm}
\subnodeIntro
\begin{itemize}
  \item \xmlNode{state}, \xmlDesc{float, required parameter}, probability for outcome 1
  \begin{itemize}
          \item \xmlAttr{outcome}, \xmlDesc{float, required parameter}, outcome value.
  \end{itemize}
  \item \xmlNode{state}, \xmlDesc{float, required parameter}, probability for outcome 2
  \begin{itemize}
          \item \xmlAttr{outcome}, \xmlDesc{float, required parameter}, outcome value.
  \end{itemize}
  \item ...
  \item \xmlNode{state}, \xmlDesc{float, required parameter}, probability for outcome K
  \begin{itemize}
          \item \xmlAttr{outcome}, \xmlDesc{float, required parameter}, outcome value.
  \end{itemize}
 \end{itemize}
Example:
\begin{lstlisting}[style=XML]
<Distributions>
  ...
    <Categorical name='testCategorical'>
        <state outcome="10">0.1</state>
        <state outcome="20">0.2</state>
        <state outcome="50">0.15</state>
        <state outcome="60">0.4</state>
        <state outcome="90">0.15</state>
    </Categorical>
  ...
</Distributions>
\end{lstlisting}

\paragraph{Uniform Discrete Distribution}
\label{subsec:UniformDiscrete}

The \textbf{UniformDiscrete} distribution is a discrete distribution which describes a random variable 
that can have $N$ values having equal probability value.
This distribution allows the user to choose two kinds of sampling strategies: with or without replacement.
In case the ``without replacement'' strategy is used, the distribution samples from the set of specified $N$ values 
reduced by the previously sampled values. 
After, the sampler has generated values for all variables, the distribution is 
resetted (i.e., the set of values that can be sampled is returned to $N$).
In case the ``with replacement'' strategy is used, the distribution samples always from the complete set of specified $N$ values.

\specBlock{a}{Uniform Discrete}
%
\attrIntro
\vspace{-5mm}
\begin{itemize}
  \itemsep0em
  \item \nameDescription
\end{itemize}
\vspace{-5mm}
\subnodeIntro
\begin{itemize}
  \item \xmlNode{lowerBound}, \xmlDesc{float, required parameter}, lower bound.
  \item \xmlNode{upperBound}, \xmlDesc{float, required parameter}, upper bound.
  \item \xmlNode{nPoints},    \xmlDesc{integer, optional parameter}, number of points between lower and upper bound
  \item \xmlNode{strategy},   \xmlDesc{string,  required parameter}, type of sampling strategy 
  (withReplacement or withoutReplacement).
  %
\end{itemize}

Example:
\begin{lstlisting}[style=XML]
<Distributions>
  ...
    <UniformDiscrete name="UD_dist">
      <lowerBound>3</lowerBound>
      <upperBound>8</upperBound>
      <strategy>orderedWithReplacement</strategy>
    </UniformDiscrete>
  ...
</Distributions>
\end{lstlisting}

\paragraph{Markov Categorical Distribution}
\label{subsec:markovCategorical}

The \textbf{MarkovCategorical} distribution is a specific discrete categorical distribution describes
a random variable that can have $K$ possible outcomes, based on the steady state probabilities provided by
Markov model.
%
\begin{itemize}
  \item \xmlNode{transition}, \xmlDesc{float, optional field}, the transition matrix of given Markov model.
  \item \xmlNode{dataFile}, \xmlDesc{string, optional xml node}. The path for the given data file, i.e. the transition matrix.
    In this node, the following attribute should be specified: 
    \begin{itemize}
      \item \xmlAttr{fileType}, \xmlDesc{string, optional field}, the type of given data file, default is `csv'.
    \end{itemize}
  \nb Either \xmlNode{transition} or \xmlNode{dataFile} is required to provide the transition matrix.
  \item \xmlNode{workingDir}, \xmlDesc{string, optional field}, the path of working directory
  \item \xmlNode{state}, \xmlDesc{required xml node}. The output from this state indicates
    the probability for outcome 1.
    In this node, the following attribute should be specified:
    \begin{itemize}
      \item \xmlAttr{outcome}, \xmlDesc{float, required field}, outcome value. 
      \item \xmlAttr{index}, \xmlDesc{integer, required field}, the index of steady state probabilities corresponding to the transition matrix. 
    \end{itemize}
  \item \xmlNode{state}, \xmlDesc{required xml node}. The output from this state indicates
    the probability for outcome 2.
    In this node, the following attribute should be specified:
    \begin{itemize}
      \item \xmlAttr{outcome}, \xmlDesc{float, required field}, outcome value. 
      \item \xmlAttr{index}, \xmlDesc{integer, required field}, the index of steady state probabilities corresponding to the transition matrix. 
    \end{itemize}
  \item ...
  \item \xmlNode{state}, \xmlDesc{required xml node}. The output from this state indicates
    the probability for outcome K.
    In this node, the following attribute should be specified:
    \begin{itemize}
      \item \xmlAttr{outcome}, \xmlDesc{float, required field}, outcome value. 
      \item \xmlAttr{index}, \xmlDesc{integer, required field}, the index of steady state probabilities corresponding to the transition matrix. 
    \end{itemize}

\end{itemize}

\textbf{Example:}

\begin{lstlisting}[style=XML]
<Simulation>
 ...
  <Distributions>
    ...
    <MarkovCategorical name="x_dist">
        <!--dataFile fileType='csv'>transitionFile</dataFile-->
        <transition>
            -1.1   0.8   0.7
            0.8    -1.4  0.2
            0.3    0.6   -0.9
        </transition>
        <state outcome='1' index='1'/>
        <state outcome='2' index='2'/>
        <state outcome='4' index='3'/>
    </MarkovCategorical>
    ...
  </Distributions>
 ...
</Simulation>
\end{lstlisting}



%%%%%% N-Dimensional Probability distributions
\subsection{N-Dimensional Probability Distributions}
\label{subsec:NdDist}
The group of $N$-Dimensional distributions allow the user to model stochastic dependences between parameters. Thus instead of using $N$ distributions for $N$ parameters, the user can define a single distribution lying in a $N$-Dimensional space.
The following $N$-Dimensional Probability Distributions are available within RAVEN:
\begin{itemize}
\item MultivariateNormal: Multivariate normal distribution (see Section~\ref{MultivariateNormal})
\item NDInverseWeight: ND Inverse Weight interpolation distribution (see Section~\ref{NDInverseWeight})
\item NDCartesianSpline: ND spline interpolation distribution (see Section~\ref{NDCartesianSpline})
\end{itemize}
For NDInverseWeight and NDCartesianSpline distributions, the user provides the sampled values of either CDF or PDF of the distribution. The sampled values can be scattered distributed (for NDInverseWeight) or over a cartesian grid (for NDCartesianSpline).

The user could specify, for each $N$-Dimensional distribution, the parameters of the random number generator function:
\begin{itemize}
\item \xmlNode{initialGridDisc}, \xmlDesc{positive integer, optional field}, user-defined initial grid discretization. This parameter specifies the number of discretizations that need to be performed, initially, for each Dimension in
order to find N-Dimensional coordinate that corresponds to the CDF represented by a random number (0-1);
\item \xmlNode{tolerance}, \xmlDesc{float, optional field}, user-defined tolerance in order to find the N-D coordinates corresponding to a random number. This tolerance is expressed in terms of CDF.
\end{itemize}
in the \xmlNode{samplerInit} block defined in sampler block \xmlNode{samplerInit} (see Section~\ref{sec:Samplers}).

\subsubsection{MultivariateNormal Distribution}
\label{MultivariateNormal}
the multivariate normal distribution or multivariate Gaussian distribution, is a generalization of the one-dimensional (univariate) normal distribution to higher dimensions.
The multivariate normal distribution is often used to describe, at least approximately, any set of (possibly) correlated real-valued random variables each of which clusters around a mean value.
The multivariate normal distribution of a $k$-dimensional random vector $\mathbf{x} = [x_1, x_2, …, x_k]$  can be written in the following notation:
$ \mathbf{x}\ \sim\ \mathcal{N}(\boldsymbol\mu,\, \boldsymbol\Sigma)$
with with $k$-dimensional mean vector

$\boldsymbol\mu= [E[x_1], E[x_2], …, E[x_k]]$

and $k \times k$ covariance matrix

$\boldsymbol\Sigma = [Cov[x_i,x_j]] , i=1,2,\ldots,k ; j=1,2,\ldots,k$

The probability distribution function for this distribution is the following:

$
f_{\mathbf x}(x_1,\ldots,x_k) =
\frac{1}{\sqrt{(2\pi)^k|\boldsymbol\Sigma|}}
\exp\left(-\frac{1}{2}({\mathbf x}-{\boldsymbol\mu})^\mathrm{T}{\boldsymbol\Sigma}^{-1}({\mathbf x}-{\boldsymbol\mu})
\right),
$

The specifications of this distribution must be defined within the xml block \xmlNode{MultivariateNormal}.
This XML node needs to contain the attributes:
\vspace{-5mm}
\begin{itemize}
\itemsep0em
\item \xmlAttr{name}, \xmlDesc{required string attribute}, user-defined identifier of this multivariate normal distribution.
%
\nb As with other objects, this is the name that can be used to refer to this specific entity from other input XML blocks.
\item \xmlAttr{method}, \xmlDesc{required string attribute}, defines which method is used to generate the multivariate normal distribution.
The only allowable methods are \xmlString{spline} and \xmlString{pca}.
%
\end{itemize}
\vspace{-5mm}

In RAVEN the MultivariateNormal distribution can be initialized through the following keywords:
\begin{itemize}
  \item \xmlNode{mu}, list of mean values of each dimension
  \item \xmlNode{covariance}, list of element values in the covariance matrix. There are two types of \xmlNode{covariance}, based on the \xmlAttr{type}:
  \begin{itemize}
    \item \xmlAttr{type}, \xmlDesc{string, optional field}, specifies the type of covariance, the default \xmlAttr{type} is \xmlString{abs}. Possible values for \xmlAttr{type} are \xmlString{abs} and \xmlString{rel}. \nb \xmlString{abs} indicates the covariance is a normal covariance matrix, while \xmlString{rel} indicates the covariance is a relative covariance matrix. In addition, method \xmlString{pca} can be combined with both types, and method \xmlString{spline} only accept the type \xmlString{abs}
  \end{itemize}
  \item \xmlNode{transformation}, \xmlDesc{XML node, optional field}, option to enable input parameter transformation using principal component analysis (PCA) approach. If this node is provided, PCA will be used to compute the principal components of input covariance matrix. The subnode \xmlNode{rank} is used to indicate the number of principal components that will be used for the input transformation. The content will specify one attribute:
  \begin{itemize}
    \item \xmlNode{rank}, \xmlDesc{positive integer, required field}, user-defined dimensionality reduction.
  \end{itemize}
\end{itemize}

Example:
\begin{lstlisting}[style=XML]
<Distributions>
  ...
    <MultivariateNormal name='MultivariateNormal_test' method='spline'>
        <mu>0.0 60.0</mu>
        <covariance>
	1.0 0.7
	0.7 1.0
        </covariance>
    </MultivariateNormal>
    <MultivariateNormal name='MultivariateNormal_abs' method='pca'>
        <mu>0.0 60.0</mu>
        <covariance type='abs'>
	1.0 0.7
	0.7 1.0
        </covariance>
    </MultivariateNormal>
    <MultivariateNormal name='MultivariateNormal_rel' method='pca'>
        <mu>0.0 60.0</mu>
        <covariance type='rel'>
	1.0 0.7
	0.7 1.0
        </covariance>
    </MultivariateNormal>
  ...
</Distributions>
\end{lstlisting}

In the following, we defined a distribution with a transformation node using PCA method. The number of principal components is
defined in \xmlNode{rank}. In this distribution, PCA is employed to restruct the multivariate normal distribution. In addition,
the size of uncorrelated variables is also determined by \xmlNode{rank}.

\begin{lstlisting}[style=XML]
<Distributions>
  ...
    <MultivariateNormal name='MultivariateNormal_test' method='pca'>
        <mu>0.0 10.0 20.0</mu>
        <covariance type="abs">
           1.0   0.7   -0.2
           0.7   1.0   0.4
           -0.2  0.4   1.0
        </covariance>
        <transformation>
          <rank>2</rank>
        </transformation>
    </MultivariateNormal>
  ...
</Distributions>
\end{lstlisting}

\subsubsection{NDInverseWeight Distribution}
\label{NDInverseWeight}
The NDInverseWeight distribution creates a $N$-Dimensional distribution given a set of points
scattered distributed. These points sample the PDF of the original distribution.
Distribution values (PDF or CDF) are calculated using the inverse weight
interpolation scheme.

\specBlock{a}{NDInverseWeight}
%
\attrsIntro
\vspace{-5mm}
\begin{itemize}
\itemsep0em
\item \nameDescription
\end{itemize}
\vspace{-5mm}


In RAVEN the NDInverseWeight distribution can be initialized through the following nodes:
\begin{itemize}
\item \xmlNode{p}, \xmlDesc{float, required parameter}, power parameter. Greater values of p assign greater influence to values closest to the interpolated point.
\item \xmlNode{data\textunderscore filename}, \xmlDesc{string, required parameter},  name of the data file containing scattered values (file type '.txt').
\begin{itemize}
\item \xmlAttr{type}, \xmlDesc{required string attribute},  indicates if the data in indicated file is PDF or CDF.
\end{itemize}
\item \xmlNode{working\textunderscore dir}, \xmlDesc{string, required parameter}, folder location of the data file
\end{itemize}

Example:
\begin{lstlisting}[style=XML]
<Distributions>
  ...
  <NDInverseWeight name='...'>
        <p>...</p>
        <dataFilename type='...'>...</dataFilename>
        <workingDir>...</workingDir>
  </NDInverseWeight>
  ...
</Distributions>
\end{lstlisting}

Each data entry contained in data\textunderscore filename is listed row by row and must be listed as follows:
\begin{itemize}
\item number of dimensions
\item number of sampled points
\item ND coordinate of each sampled point
\item value of each sampled point
\end{itemize}

As an example, the following shows the data entries contained in data\textunderscore filename for a 3-dimensional data set that contained two sampled CDF values: ([0.0,0.0,0.0], 0.1) and ([1.0, 1.0,0.0], 0.8)

Example scattered data file:
\begin{lstlisting}
3
2
0.0
0.0
0.0
1.0
1.0
0.0
0.1
0.8
\end{lstlisting}

\subsubsection{NDCartesianSpline Distribution}
\label{NDCartesianSpline}

The NDCartesianSpline distribution creates a $N$-Dimensional distribution given a set of points
regularly distributed on a cartesian grid. These points sample the PDF of the original distribution.
Distribution values (PDF or CDF) are calculated using the ND spline
interpolation scheme.


\specBlock{a}{NDCartesianSpline}
%
\attrsIntro
\vspace{-5mm}
\begin{itemize}
\itemsep0em
\item \nameDescription
\end{itemize}
\vspace{-5mm}


In RAVEN the NDCartesianSpline distribution can be initialized through the following nodes:
\begin{itemize}
\item \xmlNode{data\textunderscore filename}, \xmlDesc{string, required parameter},  name of the data file containing scattered values (file type '.txt').
\begin{itemize}
\item \xmlAttr{type}, \xmlDesc{required string attribute}, indicates if the data in indicated file is PDF or CDF.
\end{itemize}
\item \xmlNode{working\textunderscore dir}, \xmlDesc{string, required parameter}, folder location of the data file
\end{itemize}

Example:
\begin{lstlisting}[style=XML]
<Distributions>
  ...
  <NDCartesianSpline name='...'>
        <dataFilename type='...'>...</dataFilename>
        <workingDir></workingDir>
  </NDCartesianSpline>
  ...
</Distributions>
\end{lstlisting}

Each data entry contained in data \textunderscore filename is listed row by row and must be listed as follows:
\begin{itemize}
\item number of dimensions
\item number of discretization for each dimension
\item discretization values for each dimension
\item value of each sampled point
\end{itemize}

As an example, the following shows the data entries contained in data \textunderscore filename for a 2-dimensional CDF data set on the following grid $(x,y)$:
\begin{itemize}
\item first dimension (x): -0.5, 0.5
\item first dimension (y): 1.0 2.0 3.0
\end{itemize}

Example scattered data file:
\begin{lstlisting}
2
2
3
-0.5
0.5
1.0
2.0
3.0
CDF value of (-0.5,1.0)
CDF value of (+0.5,1.0)
CDF value of (-0.5,2.0)
CDF value of (+0.5,2.0)
CDF value of (-0.5,3.0)
CDF value of (+0.5,3.0)
\end{lstlisting}
